
% Default to the notebook output style

    


% Inherit from the specified cell style.




    
\documentclass[11pt]{article}

    
    
    \usepackage[T1]{fontenc}
    % Nicer default font (+ math font) than Computer Modern for most use cases
    \usepackage{mathpazo}

    % Basic figure setup, for now with no caption control since it's done
    % automatically by Pandoc (which extracts ![](path) syntax from Markdown).
    \usepackage{graphicx}
    % We will generate all images so they have a width \maxwidth. This means
    % that they will get their normal width if they fit onto the page, but
    % are scaled down if they would overflow the margins.
    \makeatletter
    \def\maxwidth{\ifdim\Gin@nat@width>\linewidth\linewidth
    \else\Gin@nat@width\fi}
    \makeatother
    \let\Oldincludegraphics\includegraphics
    % Set max figure width to be 80% of text width, for now hardcoded.
    \renewcommand{\includegraphics}[1]{\Oldincludegraphics[width=.8\maxwidth]{#1}}
    % Ensure that by default, figures have no caption (until we provide a
    % proper Figure object with a Caption API and a way to capture that
    % in the conversion process - todo).
    \usepackage{caption}
    \DeclareCaptionLabelFormat{nolabel}{}
    \captionsetup{labelformat=nolabel}

    \usepackage{adjustbox} % Used to constrain images to a maximum size 
    \usepackage{xcolor} % Allow colors to be defined
    \usepackage{enumerate} % Needed for markdown enumerations to work
    \usepackage{geometry} % Used to adjust the document margins
    \usepackage{amsmath} % Equations
    \usepackage{amssymb} % Equations
    \usepackage{textcomp} % defines textquotesingle
    % Hack from http://tex.stackexchange.com/a/47451/13684:
    \AtBeginDocument{%
        \def\PYZsq{\textquotesingle}% Upright quotes in Pygmentized code
    }
    \usepackage{upquote} % Upright quotes for verbatim code
    \usepackage{eurosym} % defines \euro
    \usepackage[mathletters]{ucs} % Extended unicode (utf-8) support
    \usepackage[utf8x]{inputenc} % Allow utf-8 characters in the tex document
    \usepackage{fancyvrb} % verbatim replacement that allows latex
    \usepackage{grffile} % extends the file name processing of package graphics 
                         % to support a larger range 
    % The hyperref package gives us a pdf with properly built
    % internal navigation ('pdf bookmarks' for the table of contents,
    % internal cross-reference links, web links for URLs, etc.)
    \usepackage{hyperref}
    \usepackage{longtable} % longtable support required by pandoc >1.10
    \usepackage{booktabs}  % table support for pandoc > 1.12.2
    \usepackage[inline]{enumitem} % IRkernel/repr support (it uses the enumerate* environment)
    \usepackage[normalem]{ulem} % ulem is needed to support strikethroughs (\sout)
                                % normalem makes italics be italics, not underlines
    

    
    
    % Colors for the hyperref package
    \definecolor{urlcolor}{rgb}{0,.145,.698}
    \definecolor{linkcolor}{rgb}{.71,0.21,0.01}
    \definecolor{citecolor}{rgb}{.12,.54,.11}

    % ANSI colors
    \definecolor{ansi-black}{HTML}{3E424D}
    \definecolor{ansi-black-intense}{HTML}{282C36}
    \definecolor{ansi-red}{HTML}{E75C58}
    \definecolor{ansi-red-intense}{HTML}{B22B31}
    \definecolor{ansi-green}{HTML}{00A250}
    \definecolor{ansi-green-intense}{HTML}{007427}
    \definecolor{ansi-yellow}{HTML}{DDB62B}
    \definecolor{ansi-yellow-intense}{HTML}{B27D12}
    \definecolor{ansi-blue}{HTML}{208FFB}
    \definecolor{ansi-blue-intense}{HTML}{0065CA}
    \definecolor{ansi-magenta}{HTML}{D160C4}
    \definecolor{ansi-magenta-intense}{HTML}{A03196}
    \definecolor{ansi-cyan}{HTML}{60C6C8}
    \definecolor{ansi-cyan-intense}{HTML}{258F8F}
    \definecolor{ansi-white}{HTML}{C5C1B4}
    \definecolor{ansi-white-intense}{HTML}{A1A6B2}

    % commands and environments needed by pandoc snippets
    % extracted from the output of `pandoc -s`
    \providecommand{\tightlist}{%
      \setlength{\itemsep}{0pt}\setlength{\parskip}{0pt}}
    \DefineVerbatimEnvironment{Highlighting}{Verbatim}{commandchars=\\\{\}}
    % Add ',fontsize=\small' for more characters per line
    \newenvironment{Shaded}{}{}
    \newcommand{\KeywordTok}[1]{\textcolor[rgb]{0.00,0.44,0.13}{\textbf{{#1}}}}
    \newcommand{\DataTypeTok}[1]{\textcolor[rgb]{0.56,0.13,0.00}{{#1}}}
    \newcommand{\DecValTok}[1]{\textcolor[rgb]{0.25,0.63,0.44}{{#1}}}
    \newcommand{\BaseNTok}[1]{\textcolor[rgb]{0.25,0.63,0.44}{{#1}}}
    \newcommand{\FloatTok}[1]{\textcolor[rgb]{0.25,0.63,0.44}{{#1}}}
    \newcommand{\CharTok}[1]{\textcolor[rgb]{0.25,0.44,0.63}{{#1}}}
    \newcommand{\StringTok}[1]{\textcolor[rgb]{0.25,0.44,0.63}{{#1}}}
    \newcommand{\CommentTok}[1]{\textcolor[rgb]{0.38,0.63,0.69}{\textit{{#1}}}}
    \newcommand{\OtherTok}[1]{\textcolor[rgb]{0.00,0.44,0.13}{{#1}}}
    \newcommand{\AlertTok}[1]{\textcolor[rgb]{1.00,0.00,0.00}{\textbf{{#1}}}}
    \newcommand{\FunctionTok}[1]{\textcolor[rgb]{0.02,0.16,0.49}{{#1}}}
    \newcommand{\RegionMarkerTok}[1]{{#1}}
    \newcommand{\ErrorTok}[1]{\textcolor[rgb]{1.00,0.00,0.00}{\textbf{{#1}}}}
    \newcommand{\NormalTok}[1]{{#1}}
    
    % Additional commands for more recent versions of Pandoc
    \newcommand{\ConstantTok}[1]{\textcolor[rgb]{0.53,0.00,0.00}{{#1}}}
    \newcommand{\SpecialCharTok}[1]{\textcolor[rgb]{0.25,0.44,0.63}{{#1}}}
    \newcommand{\VerbatimStringTok}[1]{\textcolor[rgb]{0.25,0.44,0.63}{{#1}}}
    \newcommand{\SpecialStringTok}[1]{\textcolor[rgb]{0.73,0.40,0.53}{{#1}}}
    \newcommand{\ImportTok}[1]{{#1}}
    \newcommand{\DocumentationTok}[1]{\textcolor[rgb]{0.73,0.13,0.13}{\textit{{#1}}}}
    \newcommand{\AnnotationTok}[1]{\textcolor[rgb]{0.38,0.63,0.69}{\textbf{\textit{{#1}}}}}
    \newcommand{\CommentVarTok}[1]{\textcolor[rgb]{0.38,0.63,0.69}{\textbf{\textit{{#1}}}}}
    \newcommand{\VariableTok}[1]{\textcolor[rgb]{0.10,0.09,0.49}{{#1}}}
    \newcommand{\ControlFlowTok}[1]{\textcolor[rgb]{0.00,0.44,0.13}{\textbf{{#1}}}}
    \newcommand{\OperatorTok}[1]{\textcolor[rgb]{0.40,0.40,0.40}{{#1}}}
    \newcommand{\BuiltInTok}[1]{{#1}}
    \newcommand{\ExtensionTok}[1]{{#1}}
    \newcommand{\PreprocessorTok}[1]{\textcolor[rgb]{0.74,0.48,0.00}{{#1}}}
    \newcommand{\AttributeTok}[1]{\textcolor[rgb]{0.49,0.56,0.16}{{#1}}}
    \newcommand{\InformationTok}[1]{\textcolor[rgb]{0.38,0.63,0.69}{\textbf{\textit{{#1}}}}}
    \newcommand{\WarningTok}[1]{\textcolor[rgb]{0.38,0.63,0.69}{\textbf{\textit{{#1}}}}}
    
    
    % Define a nice break command that doesn't care if a line doesn't already
    % exist.
    \def\br{\hspace*{\fill} \\* }
    % Math Jax compatability definitions
    \def\gt{>}
    \def\lt{<}
    % Document parameters
    \title{RdaUSACE}
    
    
    

    % Pygments definitions
    
\makeatletter
\def\PY@reset{\let\PY@it=\relax \let\PY@bf=\relax%
    \let\PY@ul=\relax \let\PY@tc=\relax%
    \let\PY@bc=\relax \let\PY@ff=\relax}
\def\PY@tok#1{\csname PY@tok@#1\endcsname}
\def\PY@toks#1+{\ifx\relax#1\empty\else%
    \PY@tok{#1}\expandafter\PY@toks\fi}
\def\PY@do#1{\PY@bc{\PY@tc{\PY@ul{%
    \PY@it{\PY@bf{\PY@ff{#1}}}}}}}
\def\PY#1#2{\PY@reset\PY@toks#1+\relax+\PY@do{#2}}

\expandafter\def\csname PY@tok@w\endcsname{\def\PY@tc##1{\textcolor[rgb]{0.73,0.73,0.73}{##1}}}
\expandafter\def\csname PY@tok@c\endcsname{\let\PY@it=\textit\def\PY@tc##1{\textcolor[rgb]{0.25,0.50,0.50}{##1}}}
\expandafter\def\csname PY@tok@cp\endcsname{\def\PY@tc##1{\textcolor[rgb]{0.74,0.48,0.00}{##1}}}
\expandafter\def\csname PY@tok@k\endcsname{\let\PY@bf=\textbf\def\PY@tc##1{\textcolor[rgb]{0.00,0.50,0.00}{##1}}}
\expandafter\def\csname PY@tok@kp\endcsname{\def\PY@tc##1{\textcolor[rgb]{0.00,0.50,0.00}{##1}}}
\expandafter\def\csname PY@tok@kt\endcsname{\def\PY@tc##1{\textcolor[rgb]{0.69,0.00,0.25}{##1}}}
\expandafter\def\csname PY@tok@o\endcsname{\def\PY@tc##1{\textcolor[rgb]{0.40,0.40,0.40}{##1}}}
\expandafter\def\csname PY@tok@ow\endcsname{\let\PY@bf=\textbf\def\PY@tc##1{\textcolor[rgb]{0.67,0.13,1.00}{##1}}}
\expandafter\def\csname PY@tok@nb\endcsname{\def\PY@tc##1{\textcolor[rgb]{0.00,0.50,0.00}{##1}}}
\expandafter\def\csname PY@tok@nf\endcsname{\def\PY@tc##1{\textcolor[rgb]{0.00,0.00,1.00}{##1}}}
\expandafter\def\csname PY@tok@nc\endcsname{\let\PY@bf=\textbf\def\PY@tc##1{\textcolor[rgb]{0.00,0.00,1.00}{##1}}}
\expandafter\def\csname PY@tok@nn\endcsname{\let\PY@bf=\textbf\def\PY@tc##1{\textcolor[rgb]{0.00,0.00,1.00}{##1}}}
\expandafter\def\csname PY@tok@ne\endcsname{\let\PY@bf=\textbf\def\PY@tc##1{\textcolor[rgb]{0.82,0.25,0.23}{##1}}}
\expandafter\def\csname PY@tok@nv\endcsname{\def\PY@tc##1{\textcolor[rgb]{0.10,0.09,0.49}{##1}}}
\expandafter\def\csname PY@tok@no\endcsname{\def\PY@tc##1{\textcolor[rgb]{0.53,0.00,0.00}{##1}}}
\expandafter\def\csname PY@tok@nl\endcsname{\def\PY@tc##1{\textcolor[rgb]{0.63,0.63,0.00}{##1}}}
\expandafter\def\csname PY@tok@ni\endcsname{\let\PY@bf=\textbf\def\PY@tc##1{\textcolor[rgb]{0.60,0.60,0.60}{##1}}}
\expandafter\def\csname PY@tok@na\endcsname{\def\PY@tc##1{\textcolor[rgb]{0.49,0.56,0.16}{##1}}}
\expandafter\def\csname PY@tok@nt\endcsname{\let\PY@bf=\textbf\def\PY@tc##1{\textcolor[rgb]{0.00,0.50,0.00}{##1}}}
\expandafter\def\csname PY@tok@nd\endcsname{\def\PY@tc##1{\textcolor[rgb]{0.67,0.13,1.00}{##1}}}
\expandafter\def\csname PY@tok@s\endcsname{\def\PY@tc##1{\textcolor[rgb]{0.73,0.13,0.13}{##1}}}
\expandafter\def\csname PY@tok@sd\endcsname{\let\PY@it=\textit\def\PY@tc##1{\textcolor[rgb]{0.73,0.13,0.13}{##1}}}
\expandafter\def\csname PY@tok@si\endcsname{\let\PY@bf=\textbf\def\PY@tc##1{\textcolor[rgb]{0.73,0.40,0.53}{##1}}}
\expandafter\def\csname PY@tok@se\endcsname{\let\PY@bf=\textbf\def\PY@tc##1{\textcolor[rgb]{0.73,0.40,0.13}{##1}}}
\expandafter\def\csname PY@tok@sr\endcsname{\def\PY@tc##1{\textcolor[rgb]{0.73,0.40,0.53}{##1}}}
\expandafter\def\csname PY@tok@ss\endcsname{\def\PY@tc##1{\textcolor[rgb]{0.10,0.09,0.49}{##1}}}
\expandafter\def\csname PY@tok@sx\endcsname{\def\PY@tc##1{\textcolor[rgb]{0.00,0.50,0.00}{##1}}}
\expandafter\def\csname PY@tok@m\endcsname{\def\PY@tc##1{\textcolor[rgb]{0.40,0.40,0.40}{##1}}}
\expandafter\def\csname PY@tok@gh\endcsname{\let\PY@bf=\textbf\def\PY@tc##1{\textcolor[rgb]{0.00,0.00,0.50}{##1}}}
\expandafter\def\csname PY@tok@gu\endcsname{\let\PY@bf=\textbf\def\PY@tc##1{\textcolor[rgb]{0.50,0.00,0.50}{##1}}}
\expandafter\def\csname PY@tok@gd\endcsname{\def\PY@tc##1{\textcolor[rgb]{0.63,0.00,0.00}{##1}}}
\expandafter\def\csname PY@tok@gi\endcsname{\def\PY@tc##1{\textcolor[rgb]{0.00,0.63,0.00}{##1}}}
\expandafter\def\csname PY@tok@gr\endcsname{\def\PY@tc##1{\textcolor[rgb]{1.00,0.00,0.00}{##1}}}
\expandafter\def\csname PY@tok@ge\endcsname{\let\PY@it=\textit}
\expandafter\def\csname PY@tok@gs\endcsname{\let\PY@bf=\textbf}
\expandafter\def\csname PY@tok@gp\endcsname{\let\PY@bf=\textbf\def\PY@tc##1{\textcolor[rgb]{0.00,0.00,0.50}{##1}}}
\expandafter\def\csname PY@tok@go\endcsname{\def\PY@tc##1{\textcolor[rgb]{0.53,0.53,0.53}{##1}}}
\expandafter\def\csname PY@tok@gt\endcsname{\def\PY@tc##1{\textcolor[rgb]{0.00,0.27,0.87}{##1}}}
\expandafter\def\csname PY@tok@err\endcsname{\def\PY@bc##1{\setlength{\fboxsep}{0pt}\fcolorbox[rgb]{1.00,0.00,0.00}{1,1,1}{\strut ##1}}}
\expandafter\def\csname PY@tok@kc\endcsname{\let\PY@bf=\textbf\def\PY@tc##1{\textcolor[rgb]{0.00,0.50,0.00}{##1}}}
\expandafter\def\csname PY@tok@kd\endcsname{\let\PY@bf=\textbf\def\PY@tc##1{\textcolor[rgb]{0.00,0.50,0.00}{##1}}}
\expandafter\def\csname PY@tok@kn\endcsname{\let\PY@bf=\textbf\def\PY@tc##1{\textcolor[rgb]{0.00,0.50,0.00}{##1}}}
\expandafter\def\csname PY@tok@kr\endcsname{\let\PY@bf=\textbf\def\PY@tc##1{\textcolor[rgb]{0.00,0.50,0.00}{##1}}}
\expandafter\def\csname PY@tok@bp\endcsname{\def\PY@tc##1{\textcolor[rgb]{0.00,0.50,0.00}{##1}}}
\expandafter\def\csname PY@tok@fm\endcsname{\def\PY@tc##1{\textcolor[rgb]{0.00,0.00,1.00}{##1}}}
\expandafter\def\csname PY@tok@vc\endcsname{\def\PY@tc##1{\textcolor[rgb]{0.10,0.09,0.49}{##1}}}
\expandafter\def\csname PY@tok@vg\endcsname{\def\PY@tc##1{\textcolor[rgb]{0.10,0.09,0.49}{##1}}}
\expandafter\def\csname PY@tok@vi\endcsname{\def\PY@tc##1{\textcolor[rgb]{0.10,0.09,0.49}{##1}}}
\expandafter\def\csname PY@tok@vm\endcsname{\def\PY@tc##1{\textcolor[rgb]{0.10,0.09,0.49}{##1}}}
\expandafter\def\csname PY@tok@sa\endcsname{\def\PY@tc##1{\textcolor[rgb]{0.73,0.13,0.13}{##1}}}
\expandafter\def\csname PY@tok@sb\endcsname{\def\PY@tc##1{\textcolor[rgb]{0.73,0.13,0.13}{##1}}}
\expandafter\def\csname PY@tok@sc\endcsname{\def\PY@tc##1{\textcolor[rgb]{0.73,0.13,0.13}{##1}}}
\expandafter\def\csname PY@tok@dl\endcsname{\def\PY@tc##1{\textcolor[rgb]{0.73,0.13,0.13}{##1}}}
\expandafter\def\csname PY@tok@s2\endcsname{\def\PY@tc##1{\textcolor[rgb]{0.73,0.13,0.13}{##1}}}
\expandafter\def\csname PY@tok@sh\endcsname{\def\PY@tc##1{\textcolor[rgb]{0.73,0.13,0.13}{##1}}}
\expandafter\def\csname PY@tok@s1\endcsname{\def\PY@tc##1{\textcolor[rgb]{0.73,0.13,0.13}{##1}}}
\expandafter\def\csname PY@tok@mb\endcsname{\def\PY@tc##1{\textcolor[rgb]{0.40,0.40,0.40}{##1}}}
\expandafter\def\csname PY@tok@mf\endcsname{\def\PY@tc##1{\textcolor[rgb]{0.40,0.40,0.40}{##1}}}
\expandafter\def\csname PY@tok@mh\endcsname{\def\PY@tc##1{\textcolor[rgb]{0.40,0.40,0.40}{##1}}}
\expandafter\def\csname PY@tok@mi\endcsname{\def\PY@tc##1{\textcolor[rgb]{0.40,0.40,0.40}{##1}}}
\expandafter\def\csname PY@tok@il\endcsname{\def\PY@tc##1{\textcolor[rgb]{0.40,0.40,0.40}{##1}}}
\expandafter\def\csname PY@tok@mo\endcsname{\def\PY@tc##1{\textcolor[rgb]{0.40,0.40,0.40}{##1}}}
\expandafter\def\csname PY@tok@ch\endcsname{\let\PY@it=\textit\def\PY@tc##1{\textcolor[rgb]{0.25,0.50,0.50}{##1}}}
\expandafter\def\csname PY@tok@cm\endcsname{\let\PY@it=\textit\def\PY@tc##1{\textcolor[rgb]{0.25,0.50,0.50}{##1}}}
\expandafter\def\csname PY@tok@cpf\endcsname{\let\PY@it=\textit\def\PY@tc##1{\textcolor[rgb]{0.25,0.50,0.50}{##1}}}
\expandafter\def\csname PY@tok@c1\endcsname{\let\PY@it=\textit\def\PY@tc##1{\textcolor[rgb]{0.25,0.50,0.50}{##1}}}
\expandafter\def\csname PY@tok@cs\endcsname{\let\PY@it=\textit\def\PY@tc##1{\textcolor[rgb]{0.25,0.50,0.50}{##1}}}

\def\PYZbs{\char`\\}
\def\PYZus{\char`\_}
\def\PYZob{\char`\{}
\def\PYZcb{\char`\}}
\def\PYZca{\char`\^}
\def\PYZam{\char`\&}
\def\PYZlt{\char`\<}
\def\PYZgt{\char`\>}
\def\PYZsh{\char`\#}
\def\PYZpc{\char`\%}
\def\PYZdl{\char`\$}
\def\PYZhy{\char`\-}
\def\PYZsq{\char`\'}
\def\PYZdq{\char`\"}
\def\PYZti{\char`\~}
% for compatibility with earlier versions
\def\PYZat{@}
\def\PYZlb{[}
\def\PYZrb{]}
\makeatother


    % Exact colors from NB
    \definecolor{incolor}{rgb}{0.0, 0.0, 0.5}
    \definecolor{outcolor}{rgb}{0.545, 0.0, 0.0}



    
    % Prevent overflowing lines due to hard-to-break entities
    \sloppy 
    % Setup hyperref package
    \hypersetup{
      breaklinks=true,  % so long urls are correctly broken across lines
      colorlinks=true,
      urlcolor=urlcolor,
      linkcolor=linkcolor,
      citecolor=citecolor,
      }
    % Slightly bigger margins than the latex defaults
    
    \geometry{verbose,tmargin=1in,bmargin=1in,lmargin=1in,rmargin=1in}
    
    

    \begin{document}
    
    
    \maketitle
    
    

    
    \hypertarget{nouxe7uxf5es-buxe1sicas-de-r}{%
\subsubsection{Noções básicas de R}\label{nouxe7uxf5es-buxe1sicas-de-r}}

\begin{center}\rule{0.5\linewidth}{\linethickness}\end{center}

\hypertarget{ide-r}{%
\paragraph{IDE R}\label{ide-r}}

\hypertarget{outras-maneiras-de-usar}{%
\paragraph{Outras maneiras de usar}\label{outras-maneiras-de-usar}}

\hypertarget{jupyter-notebook}{%
\subsubsection{Jupyter Notebook}\label{jupyter-notebook}}

\emph{O R ao final da primeira palavra é de R!}

Rodar uma linha:

\begin{verbatim}
CTRL+ENTER
\end{verbatim}

\emph{Como aqui estamos dentro do Jupyter Notebook, basta apertar o
botão Run na barra de ferramentas. Mas o CRTL+ENTER também funciona!}

Comentar várias:

\begin{verbatim}
CTRL+C #
\end{verbatim}

Comece criando um novo ambiente, colocando pastas para data, scripts e
output

\emph{As janelinhas com In {[} {]} representam uma máquina virtual em
cada uma. Como a linguagem padrão setada para este Jupyter Notebook foi
R, elas rodam o interpretador R}

É possível mudar algumas delas para outras linguagens, como Python ou
Java. Para algumas linguagens, como FORTRAN, é necessário:

\begin{enumerate}
\def\labelenumi{\arabic{enumi}.}
\item
  abrir a janela de comandos do Windows clicando com o botão
  \textbf{direito} e indicando \textbf{Executar como Administrador} (é
  preciso ter privilégios de administrador);
\item
  fazer a instalação com o comando

  conda install \ldots{}
\end{enumerate}

Ou

\begin{enumerate}
\def\labelenumi{\arabic{enumi}.}
\tightlist
\item
  abrir o programa \textbf{Anaconda Navigator} e procurar por nome, pelo
  pacote da nova linguagem que deseja habilitar. A instalação já cuida
  de todas as dependências
\end{enumerate}

    Invocando o help:

\emph{Isso é meio universal. Serve para funções, classes, objetos,
pacotes\ldots{}}

    \begin{Verbatim}[commandchars=\\\{\}]
{\color{incolor}In [{\color{incolor}43}]:} \PY{o}{?}\PY{k+kp}{abs}
         
         a \PY{o}{\PYZlt{}\PYZhy{}} \PY{l+m}{3}
         
         \PY{k+kp}{print}\PY{p}{(}a\PY{p}{)}
\end{Verbatim}


    \begin{Verbatim}[commandchars=\\\{\}]
[1] 3

    \end{Verbatim}

    \hypertarget{instalando-e-usando-pacotes}{%
\paragraph{Instalando e usando
pacotes}\label{instalando-e-usando-pacotes}}

Um pacote é uma biblioteca de classes que adicionam funcionalidades ao
R. Eles não vêm instalados ``de fábrica'' para não deixar o R pesado
demais, com coisas que talvez você não vá usar nunca. É importante antes
de chamar o pacote (com o comando \textbf{library}) realizar sua
instalação

\emph{Se você fica confuso com os comandos abaixo, ou achar que precisa
aprender mais, existem dois lugares bons para googlar antes de se
apavorar}

Eles são:

\begin{enumerate}
\def\labelenumi{\arabic{enumi}.}
\tightlist
\item
  Datacamp - a Datacamp é uma plataforma americana e que visa lucro. Ela
  funciona mais ou menos como a Udemy. As pessoas criam seus cursos lá e
  outras podem comprar. Uma prática para atrair clientes é
  disponibilizar gratuitamente milhares de tutoriais em TI, muitos deles
  sobre R. Então é possível aprender muito sem gastar um tostão. Se você
  se interessar, poderá contratar mais tarde outros tutoriais mais
  avançados a preços módicos. Para quem está iniciando, todos são
  gratuitos!
\end{enumerate}

Se achar confuso ou incompleto o que está escrito abaixo, experimente o
seguinte
\href{https://www.datacamp.com/community/tutorials/r-packages-guide}{tutorial}

\begin{enumerate}
\def\labelenumi{\arabic{enumi}.}
\setcounter{enumi}{1}
\tightlist
\item
  Stackoverflow - é o maior repositório de boas respostas para
  questionamentos em todas as linguagens de programação. Existe uma
  versão mais limitada em português. Ao googlar, se encontrar um
  direcionamento para a Stack, provavelmente terá a resposta para a sua
  questão de uma maneira direta
\end{enumerate}

Por exemplo, esse cara teve um problema instalando o RCurl. Foi
solicitado que se passasse o caminho explicitamente. A pergunta e a
solução encontram-se
\href{https://stackoverflow.com/questions/42459423/cannot-install-r-packages-in-jupyter-notebook}{aqui}

\begin{center}\rule{0.5\linewidth}{\linethickness}\end{center}

Instalando um pacote:

\emph{Observe bem a mensagem apontada após executar o comando. Ela pode
informar diversas coisas interessantes!}

    \begin{Verbatim}[commandchars=\\\{\}]
{\color{incolor}In [{\color{incolor}11}]:} install.packages\PY{p}{(}\PY{l+s}{\PYZdq{}}\PY{l+s}{readxl\PYZdq{}}\PY{p}{)}
\end{Verbatim}


    \begin{Verbatim}[commandchars=\\\{\}]
Warning message:
"package 'readxl' is in use and will not be installed"
    \end{Verbatim}

    Rodar um pacote:

    \begin{Verbatim}[commandchars=\\\{\}]
{\color{incolor}In [{\color{incolor}12}]:} \PY{k+kn}{library}\PY{p}{(}readxl\PY{p}{)}
\end{Verbatim}


    \begin{Verbatim}[commandchars=\\\{\}]
{\color{incolor}In [{\color{incolor}19}]:} \PY{k+kp}{try}\PY{p}{(}readx1\PY{p}{)}
\end{Verbatim}


    \begin{Verbatim}[commandchars=\\\{\}]
{\color{incolor}In [{\color{incolor}14}]:} install.packages\PY{p}{(}\PY{l+s}{\PYZdq{}}\PY{l+s}{RCurl\PYZdq{}}\PY{p}{)}
\end{Verbatim}


    \begin{Verbatim}[commandchars=\\\{\}]
also installing the dependency 'bitops'


    \end{Verbatim}

    \begin{Verbatim}[commandchars=\\\{\}]
package 'bitops' successfully unpacked and MD5 sums checked
package 'RCurl' successfully unpacked and MD5 sums checked

The downloaded binary packages are in
	C:\textbackslash{}Users\textbackslash{}epasseto\textbackslash{}AppData\textbackslash{}Local\textbackslash{}Temp\textbackslash{}Rtmpo75EWs\textbackslash{}downloaded\_packages

    \end{Verbatim}

    \begin{center}\rule{0.5\linewidth}{\linethickness}\end{center}

Comparação:

\begin{verbatim}
==
\end{verbatim}

\emph{observe que o resultado de uma comparação será sempre TRUE ou
FALSE}

    \begin{Verbatim}[commandchars=\\\{\}]
{\color{incolor}In [{\color{incolor}8}]:} \PY{l+m}{3} \PY{o}{==} \PY{l+m}{3}
\end{Verbatim}


    TRUE

    
    \begin{Verbatim}[commandchars=\\\{\}]
{\color{incolor}In [{\color{incolor}20}]:} A \PY{o}{\PYZlt{}\PYZhy{}} \PY{l+m}{3}
         B \PY{o}{\PYZlt{}\PYZhy{}} \PY{l+m}{4}
         
         A \PY{o}{==} B
\end{Verbatim}


    FALSE

    
    \begin{center}\rule{0.5\linewidth}{\linethickness}\end{center}

Criar uma variável:

    \begin{Verbatim}[commandchars=\\\{\}]
{\color{incolor}In [{\color{incolor}22}]:} a \PY{o}{\PYZlt{}\PYZhy{}} \PY{l+m}{3}
         a
\end{Verbatim}


    3

    
    \emph{Observe que para imprimir o conteúdo de um objeto, se este for o
último comando de uma célula, não é necessário invocar explicitamente o
comando \textbf{print}. O Jupyter Notebook acredita que após rodar um
pedaço de código, seu último comando seria para exibir algo na tela. Mas
isso também pode ser feito explicitamente:}

    \begin{Verbatim}[commandchars=\\\{\}]
{\color{incolor}In [{\color{incolor}24}]:} \PY{k+kp}{print}\PY{p}{(}a\PY{p}{)}
\end{Verbatim}


    \begin{Verbatim}[commandchars=\\\{\}]
[1] 3

    \end{Verbatim}

    Tipo de dado:

\emph{Isso é útil para eu perguntar qual o tipo de dado que está gravado
em uma determinada variável. No caso de datasets, eu posso perguntar o
tipo de dado gravado em uma coluna!}

    \begin{Verbatim}[commandchars=\\\{\}]
{\color{incolor}In [{\color{incolor}28}]:} \PY{k+kp}{class}\PY{p}{(}a\PY{p}{)}
\end{Verbatim}


    'numeric'

    
    \begin{Verbatim}[commandchars=\\\{\}]
{\color{incolor}In [{\color{incolor}25}]:} \PY{k+kp}{class}\PY{p}{(}\PY{l+m}{6.78}\PY{p}{)}
\end{Verbatim}


    'numeric'

    
    \begin{Verbatim}[commandchars=\\\{\}]
{\color{incolor}In [{\color{incolor}27}]:} teste \PY{o}{\PYZlt{}\PYZhy{}} \PY{l+s}{\PYZdq{}}\PY{l+s}{Olá Mundo\PYZdq{}}
         \PY{k+kp}{class}\PY{p}{(}teste\PY{p}{)}
\end{Verbatim}


    'character'

    
    Data:

    \begin{Verbatim}[commandchars=\\\{\}]
{\color{incolor}In [{\color{incolor}30}]:} minhadata \PY{o}{\PYZlt{}\PYZhy{}} \PY{k+kp}{as.Date}\PY{p}{(}\PY{l+s}{\PYZdq{}}\PY{l+s}{2018\PYZhy{}09\PYZhy{}04\PYZdq{}}\PY{p}{)}
         minhadata
\end{Verbatim}


    2018-09-04

    
    Data, mais detalhado:

\emph{Isso aqui é um antigo padrão do UNIX e que acabou se tornando
universalizado. Então se quiser formatar um campo data-hora sem ter dor
de cabeça posteriormente, é bom ver um tutorial ou uma tabela com os
vários padrões \textbf{POSIXct} e usar isso!}

Uma documentação trazendo todos os formatos possíveis no
\textbf{POSIXct} encontra-se
\href{https://www.stat.berkeley.edu/~s133/dates.html}{aqui}

    \begin{Verbatim}[commandchars=\\\{\}]
{\color{incolor}In [{\color{incolor}31}]:} \PY{k+kp}{as.POSIXct}\PY{p}{(}\PY{l+s}{\PYZdq{}}\PY{l+s}{2018\PYZhy{}09\PYZhy{}04 08:00:00\PYZdq{}}\PY{p}{)}
\end{Verbatim}


    
    \begin{verbatim}
[1] "2018-09-04 08:00:00 -03"
    \end{verbatim}

    
    \begin{center}\rule{0.5\linewidth}{\linethickness}\end{center}

Ler csv:

    \begin{Verbatim}[commandchars=\\\{\}]
{\color{incolor}In [{\color{incolor} }]:} brazil\PYZus{}df \PY{o}{\PYZlt{}\PYZhy{}} read.csv\PY{p}{(}\PY{l+s}{\PYZdq{}}\PY{l+s}{data/brazil\PYZus{}df.csv\PYZdq{}}\PY{p}{)}
        \PY{k+kt}{data.frame}
        View\PY{p}{(}brazil\PYZus{}df\PY{p}{)} \PY{c+c1}{\PYZsh{}mostra tabela}
\end{Verbatim}


    \begin{center}\rule{0.5\linewidth}{\linethickness}\end{center}

Criar um dicionário:

\emph{Dicionários são extremamente úteis para se guardar coisas de uma
maneira organizada. Eu posso por exemplo, definir uma das colunas do meu
dataset como sendo um dicionário. Lá eu guardo todo um repositório de
informações sobre cada ítem que eu tenho cadastrado. Por exemplo, em
entradas de cadastro de livros, eu posso ter um dicionário com termos
úteis encontrados em cada livro}

Um dicionário tem uma chave e um conteúdo. É algo assim

\begin{itemize}
\tightlist
\item
  aspirina: medicamento contra indicado para o caso de dengue
\end{itemize}

Minha chave é aspirina e ao invocar esta palavra, eu obterei
``medicamento contra indicado para o caso de dengue''

Isso pode parecer banal, mas em \textbf{cálculos complexos} eu posso
ganhar muito e muito tempo computacional com o truque do dicionário.
Então eu tenho uma série de polinômios e para o valor de X = 2.31, eu
acabei de calcular qual o valor de X\^{}3. O que eu faço é guardar este
resultado num dicionário. Daí no próximo passo, eu pergunto se meu
dicionário possui a entrada ``2.31''\ldots{} oh, já possui? Então eu
simplesmente copio o valor para dentro da minha equação, sem ter que
recalcular. Ah, não possui? Então daí eu calculo e acumulo na minha
``colinha'' o resultado\ldots{}

    \begin{Verbatim}[commandchars=\\\{\}]
{\color{incolor}In [{\color{incolor}32}]:} foo \PY{o}{\PYZlt{}\PYZhy{}} \PY{k+kt}{vector}\PY{p}{(}mode\PY{o}{=}\PY{l+s}{\PYZdq{}}\PY{l+s}{list\PYZdq{}}\PY{p}{,} length\PY{o}{=}\PY{l+m}{3}\PY{p}{)}
         \PY{k+kp}{print}\PY{p}{(}foo\PY{p}{)}
         \PY{k+kp}{names}\PY{p}{(}foo\PY{p}{)} \PY{o}{\PYZlt{}\PYZhy{}} \PY{k+kt}{c}\PY{p}{(}\PY{l+s}{\PYZdq{}}\PY{l+s}{tic\PYZdq{}}\PY{p}{,} \PY{l+s}{\PYZdq{}}\PY{l+s}{tac\PYZdq{}}\PY{p}{,} \PY{l+s}{\PYZdq{}}\PY{l+s}{toe\PYZdq{}}\PY{p}{)}
         foo\PY{p}{[[}\PY{l+m}{1}\PY{p}{]]} \PY{o}{\PYZlt{}\PYZhy{}} \PY{l+m}{12}\PY{p}{;} foo\PY{p}{[[}\PY{l+m}{2}\PY{p}{]]} \PY{o}{\PYZlt{}\PYZhy{}} \PY{l+m}{22}\PY{p}{;} foo\PY{p}{[[}\PY{l+m}{3}\PY{p}{]]} \PY{o}{\PYZlt{}\PYZhy{}} \PY{l+m}{33}
         \PY{k+kp}{print}\PY{p}{(}\PY{k+kp}{names}\PY{p}{(}foo\PY{p}{)}\PY{p}{)}
\end{Verbatim}


    \begin{Verbatim}[commandchars=\\\{\}]
[[1]]
NULL

[[2]]
NULL

[[3]]
NULL

[1] "tic" "tac" "toe"

    \end{Verbatim}

    \begin{center}\rule{0.5\linewidth}{\linethickness}\end{center}

Listas:

\emph{Observe que cada linguagem de programação possui suas
peculiaridades. No Python, não é exatamente desta maneira que se cria
uma lista. Outro detalhe importante, o índice da lista no Python começa
em zero. No R começa em 1!}

Listas são extremamente importantes em linguagens de programação. Isso é
uma pilha. Eu posso ir alimentando isso a partir do primeiro ou do
último elemento e eles se mantém em uma ordem 1, 2, 3\ldots{} e eu posso
remover o primeiro ou último elemento de uma lista. Eu também posso
fazer um loop invocando todos os elementos de uma lista, o R irá me dar
cada um até esgotar a lista!

    \begin{Verbatim}[commandchars=\\\{\}]
{\color{incolor}In [{\color{incolor}40}]:} z \PY{o}{\PYZlt{}\PYZhy{}} \PY{l+m}{1}\PY{o}{:}\PY{l+m}{8}
         \PY{k+kp}{print}\PY{p}{(}\PY{k+kp}{length}\PY{p}{(}z\PY{p}{)}\PY{p}{)}
         \PY{k+kp}{print}\PY{p}{(}z\PY{p}{)}
         z\PY{p}{[}\PY{l+m}{1}\PY{p}{]}
\end{Verbatim}


    \begin{Verbatim}[commandchars=\\\{\}]
[1] 8
[1] 1 2 3 4 5 6 7 8

    \end{Verbatim}

    1

    
    \begin{center}\rule{0.5\linewidth}{\linethickness}\end{center}

Compartilhando estruturas de dados:

\begin{itemize}
\tightlist
\item
  Feather
\end{itemize}

\emph{Às vezes eu já tenho meus códigos de filtragem e correção de uma
série de dados em alguma outra linguagem de programação. Antigamente,
como uma linguagem não ``conversa'' com a outra, eu precisava
reprogramar tudo. Hoje não é mais assim!}

Suponha que eu tenha uma série de rotinas escritas em Python para buscar
planilhas Excel em um diretório específico e ler a primeira coluna de
cada uma delas. Se encontrar um determinado tipo de dado, devo puxar o
registro e adicioná-lo a um dataset. Depois disso eu farei uma série de
filtragens e correções com os dados encontrados

O Python por padrão usa uma estrutura para dados chamada Pandas. O R
infelizmente não trabalha com o Pandas. Não desejamos fazer importações
de dados. Todo mundo sabe que numa importação, erros podem aparecer e
daí todo aquele meu processo que era para ser automático fica
comprometido. Felizmente existe uma solução para isso

É necessário instalar no Jupyter Notebook o pacote Feather. O Feather lê
e grava um arquivo binário em Python, contendo meu dataset Pandas. E o
Feather lê e grava o mesmo arquivo binário em R. É desta maneira que se
trabalha, não só com R x Python mas com toda linguagem de programação!

    \begin{center}\rule{0.5\linewidth}{\linethickness}\end{center}

Limpar memória:

\emph{esse é um comando universal do R}

    \begin{Verbatim}[commandchars=\\\{\}]
{\color{incolor}In [{\color{incolor} }]:} \PY{k+kp}{rm}\PY{p}{(}doce.river.data\PY{p}{)}
\end{Verbatim}


    \begin{center}\rule{0.5\linewidth}{\linethickness}\end{center}

Factor:

\emph{É uma característica comum em muitos datasets}

Suponha que eu tenha em um cadastro as opções abaixo. Ao invés de salvar
nomes numa coluna, eu salvo o factor, ou seja, um código inteiro que
representa cada uma das variantes possíveis do meu campo. Muitas vezes
isso causa uma confusão enorme na hora de processar dados. Não adianta
googlar por ``combo box'', a palavra correta é ``factor''!

\begin{itemize}
\tightlist
\item
  transformar 1-``Male'', 2-``Female'', 3-``Not informed''
\end{itemize}

    \begin{center}\rule{0.5\linewidth}{\linethickness}\end{center}

Dados grandes:

\emph{Às vezes encontramos no R ou no Python algumas linhas de código
agindo sobre datasets com uma sintaxe similar ao SQL}

As linguagens de programação, como o R, possuem comandos especiais para
fazer substituição, vinculações, filtragem de dados em colunas de
datasets. Normalmente esses comandos costumam ser muito eficientes

No entanto, como muitos usuários de R já se acostumaram a escrever
consultas ao modo SQL, existe um pacote específico para escrever suas
consultas no próprio R, ao modo SQL. É bastante funcional e na maioria
das vezes, basta depois dar um CTRL+C e um CTRL+V naquela sua consulta
já escrita e trazê-la para dentro do R!

\begin{itemize}
\tightlist
\item
  sintaxe tipo SQL
\end{itemize}

    \begin{center}\rule{0.5\linewidth}{\linethickness}\end{center}

\hypertarget{exemplo-de-leitura-e-operauxe7uxf5es-em-dados-a-partir-de-um-.csv}{%
\paragraph{Exemplo de leitura e operações em dados a partir de um
.csv:}\label{exemplo-de-leitura-e-operauxe7uxf5es-em-dados-a-partir-de-um-.csv}}

\emph{Seguem vários exemplos de operações comuns usados o lidar com
datasets no R}

Os exemplos foram dados na classe. Recomenda-se pegar um tutorial na
Datacamp para aprofundar nesse assunto. É realmente fundamental para um
cientista de dados dominar todos os comandos mais usuais de filtragem,
busca, substituição e exibição de dados em datasets. Mais de 80\% do
trabalho se resumirá a estas operações!

Uma vez o seu dataset estando totalmente filtrado, sem nulos e
corrigido, as operações de plotagem e análises finais dos dados se
tornarão extremamente simples!

    \begin{Verbatim}[commandchars=\\\{\}]
{\color{incolor}In [{\color{incolor} }]:} \PY{k+kn}{library}\PY{p}{(}data.table\PY{p}{)}
        brazil\PYZus{}df \PY{o}{\PYZlt{}\PYZhy{}} fread\PY{p}{(}\PY{l+s}{\PYZdq{}}\PY{l+s}{data/brazil\PYZus{}df.csv\PYZdq{}}\PY{p}{)}
        \PY{k+kn}{library}\PY{p}{(}readr\PY{p}{)}
        x \PY{o}{\PYZlt{}\PYZhy{}}read\PYZus{}csv\PY{p}{(}\PY{l+s}{\PYZdq{}}\PY{l+s}{data/brazil\PYZus{}df.csv\PYZdq{}}\PY{p}{)}
        x\PY{p}{[}\PY{k+kp}{row}\PY{p}{,} column\PY{p}{]}
        x\PY{p}{[}\PY{l+m}{2}\PY{o}{:}\PY{l+m}{5}\PY{p}{,}\PY{k+kt}{c}\PY{p}{(}\PY{l+m}{1}\PY{p}{,}\PY{l+m}{3}\PY{p}{)}\PY{p}{]}
        x\PY{p}{[}\PY{l+m}{2}\PY{o}{:}\PY{l+m}{5}\PY{p}{,}\PY{k+kt}{c}\PY{p}{(}\PY{l+s}{\PYZdq{}}\PY{l+s}{date\PYZdq{}}\PY{p}{,}\PY{l+s}{\PYZdq{}}\PY{l+s}{NT\PYZdq{}}\PY{p}{)}
\end{Verbatim}


    Extrair uma coluna como vetor:

    \begin{Verbatim}[commandchars=\\\{\}]
{\color{incolor}In [{\color{incolor} }]:} flow \PY{o}{\PYZlt{}\PYZhy{}} x\PY{o}{\PYZdl{}}Q
        x\PY{p}{[[}\PY{l+s}{\PYZdq{}}\PY{l+s}{Q\PYZdq{}}\PY{p}{]]}
\end{Verbatim}


    Extrair como dataset:

    \begin{Verbatim}[commandchars=\\\{\}]
{\color{incolor}In [{\color{incolor} }]:} flow \PY{o}{\PYZlt{}\PYZhy{}}x\PY{p}{[}\PY{p}{,}\PY{l+s}{\PYZdq{}}\PY{l+s}{Q\PYZdq{}}\PY{p}{]}
        x\PY{o}{\PYZdl{}}Q\PY{p}{[}\PY{k+kt}{c}\PY{p}{(}\PY{l+m}{1}\PY{o}{:}\PY{l+m}{5}\PY{p}{,}\PY{l+m}{10}\PY{p}{)}\PY{p}{]}
\end{Verbatim}


    Lógico:

    \begin{Verbatim}[commandchars=\\\{\}]
{\color{incolor}In [{\color{incolor} }]:} x\PY{o}{\PYZdl{}}Q \PY{o}{\PYZgt{}} \PY{l+m}{400}
\end{Verbatim}


    Filtro:

    \begin{Verbatim}[commandchars=\\\{\}]
{\color{incolor}In [{\color{incolor} }]:} high\PYZus{}flow \PY{o}{\PYZlt{}\PYZhy{}} x\PY{p}{[}x\PY{o}{\PYZdl{}}Q \PY{o}{\PYZgt{}} \PY{l+m}{400}\PY{p}{,}\PY{p}{]}
        high\PYZus{}flow
\end{Verbatim}


    Flag:

    \begin{Verbatim}[commandchars=\\\{\}]
{\color{incolor}In [{\color{incolor} }]:} x\PY{o}{\PYZdl{}}status\PY{p}{[}x\PY{o}{\PYZdl{}}Q \PY{o}{\PYZgt{}} \PY{l+m}{400}\PY{p}{]} \PY{o}{\PYZlt{}\PYZhy{}} \PY{l+m}{2}
        View\PY{p}{(}x\PY{p}{)}
\end{Verbatim}


    Um jeito mais fácil:

    \begin{Verbatim}[commandchars=\\\{\}]
{\color{incolor}In [{\color{incolor} }]:} \PY{k+kn}{library}\PY{p}{(}dplyr\PY{p}{)}
        \PY{k+kn}{library}\PY{p}{(}tidyr\PY{p}{)}
\end{Verbatim}


    Concatenate (campo) - como lista:

    \begin{Verbatim}[commandchars=\\\{\}]
{\color{incolor}In [{\color{incolor} }]:} \PY{k+kt}{c}\PY{p}{(}\PY{l+s}{\PYZdq{}}\PY{l+s}{a\PYZdq{}}\PY{p}{,}\PY{l+s}{\PYZdq{}}\PY{l+s}{b\PYZdq{}}\PY{p}{)}
\end{Verbatim}


    \hypertarget{trabalhar-com-bancos-de-dados}{%
\subsubsection{Trabalhar com bancos de
dados}\label{trabalhar-com-bancos-de-dados}}

\begin{center}\rule{0.5\linewidth}{\linethickness}\end{center}

    Operações de dataset:

    \begin{Verbatim}[commandchars=\\\{\}]
{\color{incolor}In [{\color{incolor} }]:} film\PYZus{}list \PY{o}{\PYZlt{}\PYZhy{}} starwars\PY{o}{\PYZdl{}}films
        View\PY{p}{(}film\PYZus{}list\PY{p}{)}
        \PY{k+kp}{names}\PY{p}{(}film\PYZus{}list\PY{p}{)} \PY{o}{\PYZlt{}\PYZhy{}} starwars\PY{o}{\PYZdl{}}name
        View\PY{p}{(}\PY{k+kp}{names}\PY{p}{(}film\PYZus{}list\PY{p}{)}\PY{p}{)}
\end{Verbatim}


    À moda SQL: (\textbf{Tidyverse})

\emph{procure documentação específica por este nome!}

    \begin{Verbatim}[commandchars=\\\{\}]
{\color{incolor}In [{\color{incolor} }]:} stars \PY{o}{\PYZlt{}\PYZhy{}} select\PY{p}{(}starwars\PY{p}{,} name\PY{p}{,} height\PY{p}{,} mass\PY{p}{,} homeworld\PY{p}{,} species\PY{p}{)}
\end{Verbatim}


    Com Pipe:

\emph{O Pipe são esses símbolos de \textgreater{}, meio que informando
que o próximo comando também está ligado}

O R, diferente do Python e outras, é uma linguagem \textbf{não
identada}. Isso quer dizer que o interpretador R não consegue saber se
você terminou uma sequência lógica só pela identação. Assim, em caso de
loops por exemplo, eu preciso

\begin{verbatim}
{abrir uma chave e colocar o loop dentro dela}
\end{verbatim}

e tanto faz se

\begin{verbatim}
eu abro {
    
    minha chave
    
    e depois vou usando uma 
        
        concatenação que me faça sentido
    
    ou não
    
    o importante ao final é fechar

}
\end{verbatim}

Como a sintaxe SQL é baseada em um ordenamento de múltiplas linhas, para
fazer sentido no R foi criado o sistema de Pipe:

\emph{Observe também que a linha do \textbf{mutate} foi quebrada em
duas. A lógica neste caso é apenas quebrar uma linha, se quiser, depois
de uma \textbf{vírgula}. No caso, a quebra e a identação são totalmente
opcionais}

Ela é idêntica a

\begin{verbatim}
mutate(BMI = height * mass, mass = 0.5 * mass) %>%
\end{verbatim}

    \begin{Verbatim}[commandchars=\\\{\}]
{\color{incolor}In [{\color{incolor} }]:} stars \PY{o}{\PYZlt{}\PYZhy{}} starwars \PY{o}{\PYZpc{}\PYZgt{}\PYZpc{}}
            select\PY{p}{(}name\PY{p}{,} height\PY{p}{,} mass\PY{p}{,} homeworld\PY{p}{,} species\PY{p}{)} \PY{o}{\PYZpc{}\PYZgt{}\PYZpc{}}
            filter\PY{p}{(}species \PY{o}{==} \PY{l+s}{\PYZdq{}}\PY{l+s}{Droid\PYZdq{}}\PY{p}{)} \PY{o}{\PYZpc{}\PYZgt{}\PYZpc{}}
            select\PY{p}{(}species\PY{p}{)} \PY{o}{\PYZpc{}\PYZgt{}\PYZpc{}}
            mutate\PY{p}{(}BMI \PY{o}{=} height \PY{o}{*} mass\PY{p}{,}
                mass \PY{o}{=} \PY{l+m}{0.5} \PY{o}{*} mass\PY{p}{)} \PY{o}{\PYZpc{}\PYZgt{}\PYZpc{}}
            arrange\PY{p}{(}desc\PY{p}{(}BMI\PY{p}{)}\PY{p}{)}
\end{Verbatim}


    Desaninhando por um campo de lista:

    \begin{Verbatim}[commandchars=\\\{\}]
{\color{incolor}In [{\color{incolor} }]:} sw1 \PY{o}{\PYZlt{}\PYZhy{}} starwars \PY{o}{\PYZpc{}\PYZgt{}\PYZpc{}}
            select\PY{p}{(}name\PY{p}{,} films\PY{p}{,} gender\PY{p}{,} height\PY{p}{)} \PY{o}{\PYZpc{}\PYZgt{}\PYZpc{}}
            unnest\PY{p}{(}films\PY{p}{,} gender\PY{p}{)}
\end{Verbatim}


    Spread, gather:

\emph{Qual a lógica dessas sequências? Às vezes meu dataset está
elegante, mas totalmente inadequado para o trabalho que eu vou fazer.
Então eu removo alguns campos, altero outros, crio campos de categorias
e empilho/desempilho, aninho/desaninho dados até ter minha estrutura de
dados do meu jeito!}

    \begin{Verbatim}[commandchars=\\\{\}]
{\color{incolor}In [{\color{incolor} }]:} sw2 \PY{o}{\PYZlt{}\PYZhy{}} starwars \PY{o}{\PYZpc{}\PYZgt{}\PYZpc{}}
            unnest\PY{p}{(}vehicles\PY{p}{)}
            select\PY{p}{(}name\PY{p}{,} vehicles\PY{p}{,} gender\PY{p}{)} \PY{o}{\PYZpc{}\PYZgt{}\PYZpc{}}
            rename\PY{p}{(}character\PYZus{}name \PY{o}{=} name\PY{p}{)}
        
        sw\PYZus{}wide \PY{o}{\PYZlt{}\PYZhy{}} sw \PY{o}{\PYZpc{}\PYZgt{}\PYZpc{}}
            spread\PY{p}{(}film\PY{p}{,} gender\PY{p}{)}
        
        sw\PYZus{}long \PY{o}{\PYZlt{}\PYZhy{}} sw\PYZus{}wide \PY{o}{\PYZpc{}\PYZgt{}\PYZpc{}}
            gather \PY{p}{(}\PY{p}{,} films\PY{p}{)}
        
        sw\PYZus{}join \PY{o}{\PYZlt{}\PYZhy{}} sw1 \PY{o}{\PYZpc{}\PYZgt{}\PYZpc{}}
            left\PYZus{}join\PY{p}{(}sw2\PY{p}{,} by \PY{o}{=} \PY{k+kt}{c}\PY{p}{(}\PY{l+s}{\PYZdq{}}\PY{l+s}{name\PYZdq{}}\PY{o}{=}\PY{l+s}{\PYZdq{}}\PY{l+s}{character\PYZus{}name\PYZdq{}}\PY{p}{,} \PY{l+s}{\PYZdq{}}\PY{l+s}{gender\PYZdq{}}\PY{p}{)}\PY{p}{)}
        
        sw\PYZus{}summary \PY{o}{\PYZlt{}\PYZhy{}} sw\PYZus{}join \PY{o}{\PYZpc{}\PYZgt{}\PYZpc{}}
            group\PYZus{}by\PY{p}{(}films\PY{p}{)} \PY{o}{\PYZpc{}\PYZgt{}\PYZpc{}}
            summarise\PY{p}{(}mean\PYZus{}height \PY{o}{=} \PY{k+kp}{mean}\PY{p}{(}height\PY{p}{,} na.rm \PY{o}{=} \PY{k+kc}{TRUE}\PY{p}{)}\PY{p}{,}
                count \PY{o}{=} n\PY{p}{(}\PY{p}{)}\PY{p}{,}
                count\PYZus{}males \PY{o}{=} \PY{k+kp}{sum}\PY{p}{(}gender \PY{o}{==} \PY{l+s}{\PYZdq{}}\PY{l+s}{male\PYZdq{}}\PY{p}{,} na.rm \PY{o}{=} \PY{k+kc}{TRUE}\PY{p}{)}\PY{p}{,}
                count\PYZus{}females \PY{o}{=} \PY{k+kp}{sum}\PY{p}{(}gender \PY{o}{==} \PY{l+s}{\PYZdq{}}\PY{l+s}{female\PYZdq{}}\PY{p}{,} na.rm \PY{o}{=} \PY{k+kc}{TRUE}\PY{p}{)}
\end{Verbatim}


    \hypertarget{planejar-operauxe7uxf5es-em-datasets}{%
\paragraph{Planejar operações em
datasets}\label{planejar-operauxe7uxf5es-em-datasets}}

As pessos normalmente se apavoram quando percebem que receberam um
dataset, mas a estrutura dele não está adequada ao que desejamos como
resultado. Na verdade, a coisa toda é bem mais simples do que parece:

\begin{enumerate}
\def\labelenumi{\arabic{enumi}.}
\item
  primeiro tente entender a estrutura do seu dataset original. Mande
  imprimir o cabeçalho e os 20 primeiros campos, tente ver se existe
  coerência em tudo o que está lá. Veja se existem muitos campos em
  branco (NA). Se precisar, faça a contagem de todos os campos para se
  certificar
\item
  se houver mais de um dataset, verifique a lógica de cada um deles,
  individualmente. Se os dados constantes estiverem coerentes,
  provavelmente você irá obter bons resultados, independente da
  estrutura que eles apresentem
\item
  se precisar trabalhar um campo \textbf{aninhado}, desaninhe este
  campo. Transforme cada ítem da sua lista ou dicionário em um registro
  independente no seu dataset. Isso pode parecer incoerente, mas acelera
  muito as operações em datasets. Remova campos inúteis. Bole uma
  política para eliminar os NAs ou para os preencher com alguma coisa
  que faça sentido
\item
  se precisar, esboce em uma folha de papel os seus datasets. E com
  setinhas, o que você quer relacionar com o que em cada um deles.
  Consultas são como aquelas máquinas de moer cana: você entra garapa,
  passa pela máquina e sai garapa. Esboce também o que espera como
  resultado
\item
  às vezes nos atrapalhamos com \textbf{relacionamentos cruzados}. Um
  campo que aparece como coluna em um, é uma linha em outro dataset! Não
  tem muito segredo. Consulte a documentação do R e você terá respostas
  claras de como resolver isso. Um relacionamento cruzado nada mais é do
  que uma operação particular de \textbf{desempilhamento}
\item
  se os datasets ainda se mostrarem complexos demais, tente
  \textbf{empilhar} ou \textbf{desempilhar}. Às vezes mudando a
  geometria de um banco de dados, você consegue maravilhas! Leia na
  documentação como fazer essas operações. Não é incomum algo como
  \textbf{desempilhe}, \textbf{desempilhe} processe algo, relacione algo
  e \textbf{empilhe}. Não tenha medo de processar e reprocessar seus
  datasets
\item
  se a coisa estiver pesada demais, crie um dataset demonstrativo mais
  leve, ou com menos registros
\item
  evite processar muitos comandos ao mesmo tempo. Isso pode gerar código
  elegante, mas normalmente confunde os processadores de datasets. Nem
  sempre código elegante quer dizer código eficiente! Eu prefiro fazer
  do jeito mais bobo: uma operação por vez. Normalmente é melhor
  processar uma operação, por exemplo, relacionando dois datasets e do
  resultado dela, processar mais uma operação, do que fazer tudo isso de
  uma só tacada
\item
  contabilize tempos de resposta, tamanho e processamento. Faça dois
  caminhos diferentes e compare. O R é cheio de recursos de avaliação de
  tempo e memória das suas operações. Quando seus datasets ficarem
  realmente grandes, essa experiência poder \textbf{realmente} fazer a
  diferença
\item
  esboce no papel além dos seus datasets, também os seus
  \textbf{relacionamentos}. É muito fácil confundir um \textbf{left
  join} com um \textbf{inner join}. Não confie na sua memória ou na sua
  intuição, é melhor diagramar do que depois tentar encontrar a besteira
\item
  faça previsões. Se eu esperava receber 111 entradas na minha nova
  tabela e vieram 108\ldots{} opa! Alguma coisa deu errado! Nunca confie
  totalmente na sua astúcia como operador de banco de dados. Se preciso,
  crie campos auxiliares para poder juntar duas colunas e compará-las.
  Ou para contar o número de determinados objetos. Isso pode fazer toda
  a diferença!
\end{enumerate}

    \hypertarget{uso-de-pacotes-no-r}{%
\subsubsection{Uso de pacotes no R}\label{uso-de-pacotes-no-r}}

\begin{center}\rule{0.5\linewidth}{\linethickness}\end{center}

\hypertarget{o-que-uxe9-um-pacote}{%
\paragraph{O que é um pacote}\label{o-que-uxe9-um-pacote}}

A descrição aparece lá no começo deste documento. Seguem mais algumas
operações envolvendo pacotes

    Abrir um pacote:

    \begin{Verbatim}[commandchars=\\\{\}]
{\color{incolor}In [{\color{incolor} }]:} \PY{k+kn}{library}\PY{p}{(}dplyr\PY{p}{)}
\end{Verbatim}


    Abrir uma função para um único uso:

\emph{Concluída a operação, a função ou o pacote é descarregado da
memória da sua máquina}

    \begin{Verbatim}[commandchars=\\\{\}]
{\color{incolor}In [{\color{incolor} }]:} river\PYZus{}df \PY{o}{\PYZlt{}\PYZhy{}} readr\PY{o}{::}read\PYZus{}c\PY{p}{(}\PY{l+s}{\PYZdq{}}\PY{l+s}{data/brazil\PYZus{}df.csv\PYZdq{}}\PY{p}{)}
\end{Verbatim}


    Conversão de datas:

\emph{A lógica desses \%e, assm como acesso à documentação, está no
começo deste documento}

    \begin{Verbatim}[commandchars=\\\{\}]
{\color{incolor}In [{\color{incolor} }]:} other\PYZus{}day \PY{o}{\PYZlt{}\PYZhy{}} \PY{l+s}{\PYZdq{}}\PY{l+s}{3/4/18\PYZdq{}}
        other\PYZus{}day\PYZus{}date \PY{o}{\PYZlt{}\PYZhy{}} \PY{k+kp}{as.Date}\PY{p}{(}other\PYZus{}day\PY{p}{,} format \PY{o}{=} \PY{l+s}{\PYZdq{}}\PY{l+s}{\PYZpc{}e/\PYZpc{}m/\PYZpc{}\PYZpc{}y\PYZdq{}}\PY{p}{)}
\end{Verbatim}


    Formato brasileiro:

\emph{Isso serve para poder aplicar/desaplicar formatação, com
interpretação dos nomes dos meses em PT-BR}

    \begin{Verbatim}[commandchars=\\\{\}]
{\color{incolor}In [{\color{incolor} }]:} \PY{k+kp}{Sys.getlocale}\PY{p}{(}\PY{l+s}{\PYZdq{}}\PY{l+s}{LC\PYZus{}TIME\PYZdq{}}\PY{p}{)}
        \PY{k+kp}{Sys.setlocale}\PY{p}{(}\PY{k+kt}{category}\PY{o}{=}\PY{l+s}{\PYZdq{}}\PY{l+s}{LC\PYZus{}TIME\PYZdq{}}\PY{p}{,} locale\PY{o}{=}\PY{l+s}{\PYZdq{}}\PY{l+s}{Portuguese\PYZus{}Brazil.1252\PYZdq{}}\PY{p}{)}
        another\PYZus{}day \PY{o}{\PYZlt{}\PYZhy{}} \PY{l+s}{\PYZdq{}}\PY{l+s}{3 abril, 2018\PYZdq{}}
        x \PY{o}{\PYZlt{}\PYZhy{}} \PY{k+kp}{as.Date}\PY{p}{(}another\PYZus{}day\PY{p}{,} format \PY{o}{=} \PY{l+s}{\PYZdq{}}\PY{l+s}{\PYZpc{}e \PYZpc{}B, \PYZpc{}Y\PYZdq{}}\PY{p}{)}
\end{Verbatim}


    Data com Timezone:

\emph{Aquele dado colhido lá no Acre foi baseado na UTM de Brasília?
Isso aqui evita problemas com timezone}

    \begin{Verbatim}[commandchars=\\\{\}]
{\color{incolor}In [{\color{incolor} }]:} date\PYZus{}time \PY{o}{\PYZlt{}\PYZhy{}} \PY{l+s}{\PYZdq{}}\PY{l+s}{3 de Abril, 2018 13:00\PYZdq{}}
        dt \PY{o}{\PYZlt{}\PYZhy{}} \PY{k+kp}{as.POSIXct}\PY{p}{(}date\PYZus{}time\PY{p}{,} format\PY{o}{=}\PY{l+s}{\PYZdq{}}\PY{l+s}{\PYZpc{}e de \PYZpc{}b, \PYZpc{}Y \PYZpc{}H:\PYZpc{}M\PYZdq{}}\PY{p}{,} 
                         tz\PY{o}{=}\PY{l+s}{\PYZdq{}}\PY{l+s}{America/Sao\PYZus{}Paulo\PYZdq{}}\PY{p}{)}
        dt2
        \PY{k+kp}{as.numeric}\PY{p}{(}dt2\PY{p}{)}
        \PY{k+kp}{as.numeric}\PY{p}{(}dt\PY{p}{)}
        \PY{k+kp}{attr}\PY{p}{(}dt\PY{p}{,} \PY{l+s}{\PYZdq{}}\PY{l+s}{tzone\PYZdq{}}\PY{p}{)}
        OlsonNames\PY{p}{(}\PY{p}{)}
        
        dt\PYZus{}lt \PY{o}{\PYZlt{}\PYZhy{}} \PY{k+kp}{as.POSIXlt}\PY{p}{(}dt2\PY{p}{)}
        dt\PYZus{}lt\PY{o}{\PYZdl{}}year\PY{l+m}{+1900}
        dt\PYZus{}lt\PY{o}{\PYZdl{}}mon
        dt\PYZus{}lt\PY{o}{\PYZdl{}}yday
\end{Verbatim}


    Sumário estatístico de datasets:

\emph{Várias informações estatísticas, com um comando simples. Pode ser
muito útil para minha primeira aproximação}

    \begin{Verbatim}[commandchars=\\\{\}]
{\color{incolor}In [{\color{incolor} }]:} \PY{k+kn}{library}\PY{p}{(}dplyr\PY{p}{)}
        \PY{k+kp}{summary}\PY{p}{(}river\PYZus{}df\PY{p}{)}
\end{Verbatim}


    Vários recursos de análise da qualidade dos dados coletados:

\emph{Lembre-se de consultar a documentação do Skimr. Sempre que usar um
pacote no R, ou qualquer linguagem, é conveniente sempre dar uma boa
olhada na sua documentação. Às vezes um pacote pode ser incompatível com
a versão de R presente neste Notebook. O Anaconda permite mudar isso. Às
vezes ele ficou obsoleto e foi deprecado, ou mudou de nome. Às vezes o R
do Jupyter Notebook não usa exatamente a mesma biblioteca. Consultando a
documentação, mesmo um pouco de código extremamente desatualizado pode
ser colocado em ordem para funcionar}

    \begin{Verbatim}[commandchars=\\\{\}]
{\color{incolor}In [{\color{incolor} }]:} install.packages\PY{p}{(}\PY{l+s}{\PYZdq{}}\PY{l+s}{skimr\PYZdq{}}\PY{p}{)}
        \PY{k+kn}{library}\PY{p}{(}skimr\PY{p}{)}
        skim\PY{p}{(}river\PYZus{}df\PY{p}{)}
\end{Verbatim}


    Modelos lineares:

\emph{Fórmula - muito importante para qualquer pacote de modelo!}

A dica do pessoal da USACE é que no R, observe sempre muito atentamente
o que vem entre parênteses depois de invocar uma função

\begin{itemize}
\tightlist
\item
  \textbf{Primeiro detalhe}, o que chamamos de \textbf{função}
  normalmente não é apenas uma reles função. Todas essas linguagens
  modernas, inclusive o R é totalmente \textbf{orientada a objeto}
\end{itemize}

Então na verdade, quando eu estou dizendo

\begin{verbatim}
mg <- unique(...)
\end{verbatim}

De fato eu estou invocando a \textbf{classe} unique() e materializando
um \textbf{objeto} desta mesma classe que irá responder pelo nome no
caso, mg. Essas classes podem e devem ser lidadas conforme toda a teoria
de \textbf{orientação a objeto}. Eu posso invocar métodos internos, ou
criar novos métodos, muitas delas possuem diversos atributos, ou eu
posso criar novos atributos, elas suportam herança, sobrescrição,
deleção e um bocado de coisas. É legal ler sobre orientação a objeto em
R. Isso poderá não complicar, mas facilitar imensamente sua vida no
futuro!

\begin{itemize}
\item
  \textbf{Segundo detalhe}, eu passo dentro dos parênteses, os meus
  \textbf{parâmetros}. Por exemplo, (nome). Eles podem ter um valor
  padrão (nome=``Camila''). Cada parâmetro é passado na sequência
  correta e nesse caso não precisa do nome do parâmetro (``Camila,
  24,''feminino), ou em qualquer ordem, desde que explicitamente
  (nome=``Camila'', idade=24, sexo=``feminino''), sempre separados por
  vírgulas. Há alguns parâmetros opcionais, mas em alguns casos, a boa
  prática de programação me diz que em alguns casos eu assim mesmo devo
  escrevê-los (sep=``,''). Isso pode tornar a identificação de erros
  \textbf{muito mais simples} (poxa, como é que eu comi bola aqui, o
  separador deste meu csv é ``;''!
\item
  \textbf{Terceiro detalhe}, a dica da USACE. As classes dos pacotes do
  R adoram receber \textbf{fórmulas} como parâmetros. Então sempre
  esteja muito atento ao que está sendo enviado para a criação do seu
  objeto. Sua fórmula estará lá. Se você comeu bola na fórmula, seu
  modelo está destruído! Isso não é muito comum em linguagens de
  programação mais simples. Mas como a maior parte das bibliotecas do R
  serve para criação de modelos matemáticos e modelos exigem fórmulas,
  observe onde elas sempre irão aparecer: nos parênteses!
\end{itemize}

Um caso aqui:

\begin{verbatim}
model_4 <- lm(NT ~ log(Q) + date + date:Q, data = MOG)
\end{verbatim}

Onde está minha fórmula de linearização para o Modelo 4?

\begin{verbatim}
NT ~ log(Q) + date + date:Q
\end{verbatim}

    O help do Linear Models:

    \begin{Verbatim}[commandchars=\\\{\}]
{\color{incolor}In [{\color{incolor} }]:} \PY{o}{?}lm
\end{Verbatim}


    \hypertarget{trabalhos-com-datasets}{%
\subsubsection{Trabalhos com datasets}\label{trabalhos-com-datasets}}

\begin{center}\rule{0.5\linewidth}{\linethickness}\end{center}

    Primeiro exercício de filtragem - dataframe MOG:

    \begin{Verbatim}[commandchars=\\\{\}]
{\color{incolor}In [{\color{incolor} }]:} \PY{k+kp}{unique}\PY{p}{(}river\PYZus{}df\PY{o}{\PYZdl{}}CODIGO\PYZus{}ESTACAO\PY{p}{)}
        MOG \PY{o}{\PYZlt{}\PYZhy{}} river\PYZus{}df \PY{o}{\PYZpc{}\PYZgt{}\PYZpc{}}
        	filter\PY{p}{(}CODIGO\PYZus{}ESTACAO \PY{o}{==} \PY{l+s}{\PYZdq{}}\PY{l+s}{MOGU02180\PYZdq{}}\PY{p}{)}
        skim\PY{p}{(}MOG\PY{p}{)}
\end{Verbatim}


    Sintaxe SQL no R:
\href{https://stackoverflow.com/questions/37770394/similar-function-to-sql-where-clause-in-r}{exemplo}

    \begin{Verbatim}[commandchars=\\\{\}]
{\color{incolor}In [{\color{incolor} }]:} \PY{k+kn}{library}\PY{p}{(}sqldf\PY{p}{)}
        \PY{k+kp}{system.time}\PY{p}{(}sqldf\PY{p}{(}\PY{l+s}{\PYZdq{}}\PY{l+s}{SELECT * FROM temps1 WHERE country != \PYZsq{}A\PYZsq{}\PYZdq{}}\PY{p}{)}\PY{p}{)}
        
        \PY{k+kn}{library}\PY{p}{(}data.table\PY{p}{)}
        \PY{k+kp}{system.time}\PY{p}{(}setDT\PY{p}{(}temps1\PY{p}{,} key \PY{o}{=} \PY{l+s}{\PYZsq{}}\PY{l+s}{country\PYZsq{}}\PY{p}{)}\PY{p}{[}\PY{o}{!}\PY{p}{(}\PY{l+s}{\PYZdq{}}\PY{l+s}{A\PYZdq{}}\PY{p}{)}\PY{p}{]}\PY{p}{)}
\end{Verbatim}


    Modelo linear:

    \begin{Verbatim}[commandchars=\\\{\}]
{\color{incolor}In [{\color{incolor} }]:} model\PYZus{}1 \PY{o}{\PYZlt{}\PYZhy{}} lm\PY{p}{(}NT \PY{o}{\PYZti{}} Q\PY{p}{,} data \PY{o}{=} MOG\PY{p}{)}
\end{Verbatim}


    Comparação de modelos:

\emph{Observe que é muito importante a fórmula que eu aplico a cada
modelo!}

Uso do comando Predict()

    \begin{Verbatim}[commandchars=\\\{\}]
{\color{incolor}In [{\color{incolor} }]:} model\PYZus{}1 \PY{o}{\PYZlt{}\PYZhy{}} lm\PY{p}{(}NT \PY{o}{\PYZti{}} Q\PY{p}{,} data \PY{o}{=} MOG\PY{p}{)}
        rsq\PYZus{}1 \PY{o}{\PYZlt{}\PYZhy{}} \PY{k+kp}{summary}\PY{p}{(}model\PYZus{}1\PY{p}{)}\PY{o}{\PYZdl{}}r.squared
        
        summart\PY{p}{(}model\PYZus{}1\PY{p}{)}
        coefficients\PY{p}{(}model\PYZus{}1\PY{p}{)}\PY{o}{\PYZdl{}}r.squared
        
        model\PYZus{}2 \PY{o}{\PYZlt{}\PYZhy{}} lm\PY{p}{(}NT \PY{o}{\PYZti{}} Q \PY{o}{+} \PY{k+kp}{date}\PY{p}{,} data \PY{o}{=} MOG\PY{p}{)}
        rsq\PYZus{}1 \PY{o}{\PYZlt{}\PYZhy{}} \PY{k+kp}{summary}\PY{p}{(}model\PYZus{}2\PY{p}{)}\PY{o}{\PYZdl{}}r.squared
        coefficients\PY{p}{(}model\PYZus{}1\PY{p}{)}\PY{o}{\PYZdl{}}r.squared
        
        model\PYZus{}3 \PY{o}{\PYZlt{}\PYZhy{}} lm\PY{p}{(}NT \PY{o}{\PYZti{}} \PY{k+kp}{log}\PY{p}{(}Q\PY{p}{)} \PY{o}{+} \PY{k+kp}{date}\PY{p}{,} data \PY{o}{=} MOG\PY{p}{)}
        rsq\PYZus{}1 \PY{o}{\PYZlt{}\PYZhy{}} \PY{k+kp}{summary}\PY{p}{(}model\PYZus{}3\PY{p}{)}\PY{o}{\PYZdl{}}r.squared
        
        model\PYZus{}4 \PY{o}{\PYZlt{}\PYZhy{}} lm\PY{p}{(}NT \PY{o}{\PYZti{}} \PY{k+kp}{log}\PY{p}{(}Q\PY{p}{)} \PY{o}{+} date \PY{o}{+} \PY{k+kp}{date}\PY{o}{:}Q\PY{p}{,} data \PY{o}{=} MOG\PY{p}{)}
        rsq\PYZus{}1 \PY{o}{\PYZlt{}\PYZhy{}} \PY{k+kp}{summary}\PY{p}{(}model\PYZus{}4\PY{p}{)}\PY{o}{\PYZdl{}}r.squared
        plot\PY{p}{(}model\PYZus{}4\PY{p}{)}
\end{Verbatim}


    Usando o ggplot2:

\emph{Observe que aes é uma função!}

    \begin{Verbatim}[commandchars=\\\{\}]
{\color{incolor}In [{\color{incolor} }]:} \PY{k+kn}{library}\PY{p}{(}ggplot2\PY{p}{)}
        rivers\PYZus{}df \PY{o}{\PYZlt{}\PYZhy{}}readr\PY{o}{::}read\PYZus{}csv\PY{p}{(}\PY{l+s}{\PYZdq{}}\PY{l+s}{data/brazil\PYZus{}df.csv\PYZdq{}}\PY{p}{)}
        ggplot\PY{p}{(}data \PY{o}{=} rivers\PYZus{}df\PY{p}{,} mapping \PY{o}{=} aes\PY{p}{(}x\PY{o}{=}Q\PY{p}{,} y\PY{o}{=}NT\PY{p}{,} color \PY{o}{=} CODIGO\PYZus{}ESTACAO\PY{p}{)}\PY{p}{)} \PY{o}{+}
            geom\PYZus{}point\PY{p}{(}\PY{p}{)} \PY{o}{+}
            theme\PYZus{}bw\PY{p}{(}\PY{p}{)}
        rivers\PYZus{}plot\PYZus{}title \PY{o}{\PYZlt{}\PYZhy{}} rivers\PYZus{}plot \PY{o}{+} 
            ggtitle\PY{p}{(}\PY{l+s}{\PYZdq{}}\PY{l+s}{Brazil River Data\PYZdq{}}\PY{p}{)}
        rivers\PYZus{}plot\PYZus{}title
\end{Verbatim}


    \begin{Verbatim}[commandchars=\\\{\}]
{\color{incolor}In [{\color{incolor} }]:} Y \PY{o}{\PYZti{}} X
        rivers\PYZus{}plot \PY{o}{+}
            facet\PYZus{}grid\PY{p}{(}\PY{l+m}{.} \PY{o}{\PYZti{}} CODIGO\PYZus{}ESTACAO\PY{p}{,} scales \PY{o}{=} \PY{l+s}{\PYZdq{}}\PY{l+s}{free\PYZus{}x\PYZdq{}}\PY{p}{)}
\end{Verbatim}


    Teste:

\emph{Foi criado um campo categórico, separando estação seca de molhada
e com ele, os gráficos foram partidos entre as duas estações}

    \begin{Verbatim}[commandchars=\\\{\}]
{\color{incolor}In [{\color{incolor} }]:} \PY{o}{?}\PY{k+kp}{ifelse}
        \PY{k+kp}{ifelse}\PY{p}{(}\PY{l+m}{4} \PY{o}{\PYZgt{}} \PY{l+m}{5}\PY{p}{,} yes \PY{o}{=} \PY{l+s}{\PYZdq{}}\PY{l+s}{strange\PYZdq{}}\PY{p}{,} no \PY{o}{=} \PY{l+s}{\PYZdq{}}\PY{l+s}{normal\PYZdq{}}\PY{p}{)}
        x \PY{o}{\PYZlt{}\PYZhy{}} \PY{k+kp}{Sys.Date}\PY{p}{(}\PY{p}{)}
        
        dry\PYZus{}start \PY{o}{\PYZlt{}\PYZhy{}} \PY{l+m}{6} \PY{c+c1}{\PYZsh{} month number}
        dry\PYZus{}end \PY{o}{\PYZlt{}\PYZhy{}} \PY{l+m}{9} \PY{c+c1}{\PYZsh{} month number}
        
        x\PYZus{}month\PYZus{}number \PY{o}{\PYZlt{}\PYZhy{}} \PY{k+kp}{as.numeric}\PY{p}{(}\PY{k+kp}{format}\PY{p}{(}x\PY{p}{,} \PY{l+s}{\PYZdq{}}\PY{l+s}{\PYZpc{}m\PYZdq{}}\PY{p}{)}\PY{p}{)}
        \PY{k+kp}{ifelse}\PY{p}{(}x\PYZus{}month\PYZus{}number \PY{o}{\PYZgt{}=} \PY{l+m}{6} \PY{o}{\PYZam{}} x\PYZus{}month\PYZus{}number \PY{o}{\PYZlt{}=} \PY{l+m}{9}\PY{p}{,}
            yes \PY{o}{=} \PY{l+s}{\PYZdq{}}\PY{l+s}{dry\PYZdq{}}\PY{p}{,} no \PY{o}{=} \PY{l+s}{\PYZdq{}}\PY{l+s}{wet\PYZdq{}}\PY{p}{)}
        
        river\PYZus{}season \PY{o}{\PYZlt{}\PYZhy{}} rivers\PYZus{}df \PY{o}{\PYZpc{}\PYZgt{}\PYZpc{}}
            mutate\PY{p}{(}month\PYZus{}number \PY{o}{=} \PY{k+kp}{as.numeric}\PY{p}{(}\PY{k+kp}{format}\PY{p}{(}\PY{k+kp}{date}\PY{p}{,} \PY{l+s}{\PYZdq{}}\PY{l+s}{\PYZpc{}m\PYZdq{}}\PY{p}{)}\PY{p}{)}\PY{p}{,}
                season \PY{o}{=} \PY{k+kp}{ifelse}\PY{p}{(}x\PYZus{}month\PYZus{}number \PY{o}{\PYZgt{}=} \PY{l+m}{6} \PY{o}{\PYZam{}} x\PYZus{}month\PYZus{}number \PY{o}{\PYZlt{}=} \PY{l+m}{9}\PY{p}{,}
                yes \PY{o}{=} \PY{l+s}{\PYZdq{}}\PY{l+s}{dry\PYZdq{}}\PY{p}{,} no \PY{o}{=} \PY{l+s}{\PYZdq{}}\PY{l+s}{wet\PYZdq{}}\PY{p}{)}\PY{p}{)}
        
        season\PYZus{}plot \PY{o}{\PYZlt{}\PYZhy{}} ggplot\PY{p}{(}river\PYZus{}season\PY{p}{,}
                              aes\PY{p}{(}x\PY{o}{=}Q\PY{p}{,} y\PY{o}{=}NT\PY{p}{,} color\PY{o}{=}CODIGO\PYZus{}ESTACAO\PY{p}{)}\PY{p}{)} \PY{o}{+}
        geom\PYZus{}point\PY{p}{(}\PY{p}{)} \PY{o}{+}
            facet\PYZus{}grid\PY{p}{(}season \PY{o}{\PYZti{}} CODIGO\PYZus{}ESTACAO\PY{p}{,} scales\PY{o}{=}\PY{l+s}{\PYZdq{}}\PY{l+s}{free\PYZus{}x\PYZdq{}}\PY{p}{)}
        season\PYZus{}plot
\end{Verbatim}


    Indicação do erro padrão:

\emph{Para fórmula é necessário definir os eixos x e y}

    \begin{Verbatim}[commandchars=\\\{\}]
{\color{incolor}In [{\color{incolor} }]:} season\PYZus{}plot \PY{o}{+}
            geom\PYZus{}smooth\PY{p}{(}method \PY{o}{=} \PY{l+s}{\PYZdq{}}\PY{l+s}{lm\PYZdq{}}\PY{p}{,} formula \PY{o}{=} y \PY{o}{\PYZti{}} x\PY{p}{)}
\end{Verbatim}


    Para desativar:

    \begin{Verbatim}[commandchars=\\\{\}]
{\color{incolor}In [{\color{incolor}42}]:} geom\PYZus{}smooth\PY{p}{(}method \PY{o}{=} \PY{l+s}{\PYZdq{}}\PY{l+s}{lm\PYZdq{}}\PY{p}{,} formula \PY{o}{=} y \PY{o}{\PYZti{}}x\PY{p}{,} se \PY{o}{=} \PY{k+kc}{FALSE}\PY{p}{)}
\end{Verbatim}


    \begin{Verbatim}[commandchars=\\\{\}]

        Error in geom\_smooth(method = "lm", formula = y \textasciitilde{} x, se = FALSE): não foi possível encontrar a função "geom\_smooth"
    Traceback:
    

    \end{Verbatim}

    gráfico de caixas:

    \begin{Verbatim}[commandchars=\\\{\}]
{\color{incolor}In [{\color{incolor} }]:} boxplot\PYZus{}q \PY{o}{\PYZlt{}\PYZhy{}} ggplot\PY{p}{(}river\PYZus{}season\PY{p}{,} aes\PY{p}{(}x\PY{o}{=}CODIGO\PYZus{}ESTACAO\PY{p}{,} y\PY{o}{=}Q\PY{p}{)}\PY{p}{)} \PY{o}{+}
            geom\PYZus{}boxplot\PY{p}{(}\PY{p}{)}
\end{Verbatim}


    redimensionando e colocando legendas:

    \begin{Verbatim}[commandchars=\\\{\}]
{\color{incolor}In [{\color{incolor} }]:} boxplot\PYZus{}q \PY{o}{+} facet\PYZus{}grid\PY{p}{(}\PY{l+m}{.} \PY{o}{\PYZti{}} CODIGO\PYZus{}ESTACAO\PY{p}{,} scales\PY{o}{=}\PY{l+s}{\PYZdq{}}\PY{l+s}{free\PYZdq{}}\PY{p}{)}
\end{Verbatim}


    escala log 10:

\emph{cuidado ao plotar pontos com log(0)!}

    \begin{Verbatim}[commandchars=\\\{\}]
{\color{incolor}In [{\color{incolor} }]:} boxplot\PYZus{}q \PY{o}{+} scale\PYZus{}y\PYZus{}log10\PY{p}{(}\PY{p}{)}
\end{Verbatim}


    estações coloridas e separadas:

    \begin{Verbatim}[commandchars=\\\{\}]
{\color{incolor}In [{\color{incolor} }]:} boxplot\PYZus{}q\PYZus{}season \PY{o}{\PYZlt{}\PYZhy{}} ggplot\PY{p}{(}river\PYZus{}season\PY{p}{,} 
                            aes\PY{p}{(}x\PY{o}{=}CODIGO\PYZus{}ESTACAO\PY{p}{,} 
                                y\PY{o}{=}Q\PY{p}{,}
                                fill\PY{o}{=}season\PY{p}{)}\PY{p}{)} \PY{o}{+}
        geom\PYZus{}boxplot\PY{p}{(}\PY{p}{)} \PY{o}{+}
        scale\PYZus{}y\PYZus{}log10\PY{p}{(}\PY{p}{)}
\end{Verbatim}


    Salvar o gráfico em pdf:

    \begin{Verbatim}[commandchars=\\\{\}]
{\color{incolor}In [{\color{incolor} }]:} ggsave\PY{p}{(}filename \PY{o}{=} \PY{l+s}{\PYZdq{}}\PY{l+s}{output/discharge\PYZus{}boxplot.pdf\PYZdq{}}\PY{p}{,}
            boxplot\PYZus{}q\PYZus{}season\PY{p}{,}
            height \PY{o}{=} \PY{l+m}{5}\PY{p}{,} width \PY{o}{=} \PY{l+m}{7}\PY{p}{)}
\end{Verbatim}


    Criando um loop:

    \begin{Verbatim}[commandchars=\\\{\}]
{\color{incolor}In [{\color{incolor} }]:} \PY{k+kr}{for} \PY{p}{(}station \PY{k+kr}{in} station\PYZus{}codes\PY{p}{)} \PY{p}{\PYZob{}}
            \PY{k+kp}{print}\PY{p}{(}station\PY{p}{)}
        \PY{p}{\PYZcb{}}
\end{Verbatim}


    Filtrando com loop por estação, plotando e salvando em pdf:

\emph{Select para colunas e filter para linhas}

    \begin{Verbatim}[commandchars=\\\{\}]
{\color{incolor}In [{\color{incolor} }]:} \PY{k+kr}{for} \PY{p}{(}station \PY{k+kr}{in} station\PYZus{}codes\PY{p}{)} \PY{p}{\PYZob{}}
          \PY{k+kp}{print}\PY{p}{(}station\PY{p}{)}
          current\PYZus{}data \PY{o}{\PYZlt{}\PYZhy{}} filter\PY{p}{(}rivers\PYZus{}df\PY{p}{,} CODIGO\PYZus{}ESTACAO \PY{o}{==} station\PY{p}{)}
          station\PYZus{}model \PY{o}{\PYZlt{}\PYZhy{}} lm\PY{p}{(}NT \PY{o}{\PYZti{}}\PY{k+kp}{log}\PY{p}{(}Q\PY{p}{)} \PY{o}{+} \PY{k+kp}{date}\PY{p}{,} data \PY{o}{=} current\PYZus{}data\PY{p}{)}
          current\PYZus{}data \PY{o}{\PYZlt{}\PYZhy{}} current\PYZus{}data \PY{o}{\PYZpc{}\PYZgt{}\PYZpc{}}
            mutate\PY{p}{(}preds \PY{o}{=} predict\PY{p}{(}station\PYZus{}model\PY{p}{)}\PY{p}{)}
            \PY{c+c1}{\PYZsh{} NT = 1.2*Q + 0.5}
          station\PYZus{}plot \PY{o}{\PYZlt{}\PYZhy{}} ggplot\PY{p}{(}current\PYZus{}data\PY{p}{,} aes\PY{p}{(}x\PY{o}{=}NT\PY{p}{,} y\PY{o}{=}preds\PY{p}{)}\PY{p}{)} \PY{o}{+}
            geom\PYZus{}point\PY{p}{(}\PY{p}{)} \PY{o}{+}
            ggtitle\PY{p}{(}station\PY{p}{)}
          \PY{k+kp}{print}\PY{p}{(}station\PYZus{}plot\PY{p}{)}
          station\PYZus{}filename \PY{o}{\PYZlt{}\PYZhy{}} \PY{k+kp}{paste0}\PY{p}{(}\PY{l+s}{\PYZdq{}}\PY{l+s}{output/preds\PYZus{}obs\PYZdq{}}\PY{p}{,} station\PY{p}{,} \PY{l+s}{\PYZdq{}}\PY{l+s}{.pdf\PYZdq{}}\PY{p}{)}
          ggsave\PY{p}{(}station\PYZus{}filename\PY{p}{,} station\PYZus{}plot\PY{p}{,} height \PY{o}{=} \PY{l+m}{5}\PY{p}{,} width \PY{o}{=} \PY{l+m}{7}\PY{p}{)}
        \PY{p}{\PYZcb{}}
\end{Verbatim}


    Usando o pacote de interatividade Plotly:

    \begin{Verbatim}[commandchars=\\\{\}]
{\color{incolor}In [{\color{incolor} }]:} install.packages\PY{p}{(}\PY{l+s}{\PYZdq{}}\PY{l+s}{plotly\PYZdq{}}\PY{p}{)}
\end{Verbatim}


    Fazendo um gráfico interativo rápido:

\emph{isso é bom para HTML}

\emph{outro pacote é R Shiny}

    \begin{Verbatim}[commandchars=\\\{\}]
{\color{incolor}In [{\color{incolor} }]:} \PY{k+kn}{library}\PY{p}{(}plotly\PY{p}{)}
        
        nt\PYZus{}q\PYZus{}plot \PY{o}{\PYZlt{}\PYZhy{}} ggplot\PY{p}{(}brazil\PYZus{}df\PY{p}{,} aes\PY{p}{(}x\PY{o}{=}Q\PY{p}{,} y\PY{o}{=}NT\PY{p}{,} color\PY{o}{=}CODIGO\PYZus{}ESTACAO\PY{p}{)}\PY{p}{)} \PY{o}{+} 
            geom\PYZus{}point\PY{p}{(}\PY{p}{)}
        nt\PYZus{}q\PYZus{}plot\PYZus{}interactive \PY{o}{\PYZlt{}\PYZhy{}} ggplotly\PY{p}{(}nt\PYZus{}q\PYZus{}plot\PY{p}{)}
        nt\PYZus{}q\PYZus{}plot\PYZus{}interactive
\end{Verbatim}


    \hypertarget{inuxedcio-dos-estudos-em-egret}{%
\subsubsection{Início dos estudos em
EGRET}\label{inuxedcio-dos-estudos-em-egret}}

\begin{center}\rule{0.5\linewidth}{\linethickness}\end{center}

    Usos (com D-3):
\href{https://owi.usgs.gov/vizlab/water-use-15/\#view=USA\&category=total}{exemplo}

    Outros:

\href{https://github.com/USGS-VIZLAB}{aqui} e

\href{https://owi.usgs.gov/vizlab/hurricane-irma/}{aqui}

    EGRET - Exploration and Graphics for River Trends

\begin{center}\rule{0.5\linewidth}{\linethickness}\end{center}

\emph{É um pacote em R para análises em mudanças de qualidade de água e
vazão de rios de longo prazo. Inclui métodos de qualidade de água
(Weighted Regressions on Time, Discharge and Season - WRTDS)}

Lindsay Carr and Laura DeCicco

USGS Data Science \href{github.com/USGS-R/EGRET}{fonte}

Guia do usuário: \href{pubs.usgs.gov/tm/04/a10}{aqui}

No github: \href{usgs-r.github.io/EGRET}{aqui}

Geral da USGS:

\begin{verbatim}
github.com/USGS-R/...
\end{verbatim}

Exemplo:

    \begin{Verbatim}[commandchars=\\\{\}]
{\color{incolor}In [{\color{incolor} }]:} github.com\PY{o}{/}USGS\PY{o}{\PYZhy{}}R\PY{o}{/}WREG
\end{Verbatim}


    Entrada de dados:

    \begin{Verbatim}[commandchars=\\\{\}]
{\color{incolor}In [{\color{incolor} }]:} \PY{k+kn}{library}\PY{p}{(}readr\PY{p}{)}
        \PY{k+kn}{library}\PY{p}{(}dplyr\PY{p}{)}
\end{Verbatim}


    \begin{Verbatim}[commandchars=\\\{\}]
{\color{incolor}In [{\color{incolor} }]:} Sample \PY{o}{\PYZlt{}\PYZhy{}} read\PYZus{}csv\PY{p}{(}\PY{l+s}{\PYZdq{}}\PY{l+s}{SAPU02900.csv\PYZdq{}}\PY{p}{)} \PY{o}{\PYZpc{}\PYZgt{}\PYZpc{}}
            select\PY{p}{(}\PY{k+kp}{date}\PY{p}{,} status\PY{p}{,} NT\PY{p}{)} \PY{o}{\PYZpc{}\PYZgt{}\PYZpc{}}
            compressData\PY{p}{(}\PY{p}{)} \PY{o}{\PYZpc{}\PYZgt{}\PYZpc{}}
            populateSampleColumns\PY{p}{(}\PY{p}{)}
\end{Verbatim}


    \begin{Verbatim}[commandchars=\\\{\}]
{\color{incolor}In [{\color{incolor} }]:} Daily \PY{o}{\PYZlt{}\PYZhy{}} read\PYZus{}csv\PY{p}{(}\PY{l+s}{\PYZdq{}}\PY{l+s}{SETA04600.csv\PYZdq{}}\PY{p}{)} \PY{o}{\PYZpc{}\PYZgt{}\PYZpc{}}
            select\PY{p}{(}dateTime \PY{o}{=} \PY{k+kp}{date}\PY{p}{,} value\PY{o}{=}Q\PY{p}{)} \PY{o}{\PYZpc{}\PYZgt{}\PYZpc{}}
            populateDaily\PY{p}{(}qConvert \PY{o}{=} \PY{l+m}{1}\PY{p}{)} \PY{c+c1}{\PYZsh{}assumes Q is cms}
\end{Verbatim}


    Programático:

    \begin{Verbatim}[commandchars=\\\{\}]
{\color{incolor}In [{\color{incolor} }]:} INFO \PY{o}{\PYZlt{}\PYZhy{}} \PY{k+kt}{data.frame}\PY{p}{(}
                        Param.units \PY{o}{=} \PY{l+s}{\PYZdq{}}\PY{l+s}{mg/L\PYZdq{}}\PY{p}{,}
                        shortName \PY{o}{=} \PY{l+s}{\PYZdq{}}\PY{l+s}{SETA04600\PYZdq{}}\PY{p}{,}
                        paramShortName \PY{o}{=} \PY{l+s}{\PYZdq{}}\PY{l+s}{Total Nitrogen\PYZdq{}}\PY{p}{,}
                        drainSqKm \PY{o}{=} \PY{l+m}{100}\PY{p}{,}
                        constitAbbrev \PY{o}{=} \PY{l+s}{\PYZdq{}}\PY{l+s}{TN\PYZdq{}}\PY{p}{,}
                        staAbbrev \PY{o}{=} \PY{l+s}{\PYZdq{}}\PY{l+s}{SETA\PYZdq{}}\PY{p}{,}
                        paStart \PY{o}{=} \PY{l+m}{10}\PY{p}{,}
                        paLong \PY{o}{=} \PY{l+m}{12}\PY{p}{,}
                        stringsAsFactors \PY{o}{=} \PY{k+kc}{FALSE}\PY{p}{)}
        
        eList \PY{o}{\PYZlt{}\PYZhy{}} mergeReport\PY{p}{(}INFO\PY{p}{,} Daily\PY{p}{,} Sample\PY{p}{)}
\end{Verbatim}


    Carregando o Egret:

    \begin{Verbatim}[commandchars=\\\{\}]
{\color{incolor}In [{\color{incolor}3}]:} install.packages\PY{p}{(}\PY{l+s}{\PYZdq{}}\PY{l+s}{EGRET\PYZdq{}}\PY{p}{)}
        \PY{k+kn}{library}\PY{p}{(}EGRET\PY{p}{)}
        \PY{o}{?}\PY{o}{?}EGRET
        
        eList \PY{o}{\PYZlt{}\PYZhy{}} Choptank\PYZus{}eList
        Daily \PY{o}{\PYZlt{}\PYZhy{}} eList\PY{o}{\PYZdl{}}Daily
        Sample \PY{o}{\PYZlt{}\PYZhy{}} eList\PY{o}{\PYZdl{}}Sample
        INFO \PY{o}{\PYZlt{}\PYZhy{}} eList\PY{o}{\PYZdl{}}INFO
\end{Verbatim}


    \begin{Verbatim}[commandchars=\\\{\}]
also installing the dependencies 'dotCall64', 'spam', 'dataRetrieval', 'fields', 'truncnorm'


    \end{Verbatim}

    \begin{Verbatim}[commandchars=\\\{\}]
package 'dotCall64' successfully unpacked and MD5 sums checked
package 'spam' successfully unpacked and MD5 sums checked
package 'dataRetrieval' successfully unpacked and MD5 sums checked
package 'fields' successfully unpacked and MD5 sums checked
package 'truncnorm' successfully unpacked and MD5 sums checked
package 'EGRET' successfully unpacked and MD5 sums checked

The downloaded binary packages are in
	C:\textbackslash{}Users\textbackslash{}epasseto\textbackslash{}AppData\textbackslash{}Local\textbackslash{}Temp\textbackslash{}Rtmpo75EWs\textbackslash{}downloaded\_packages

    \end{Verbatim}

    \begin{Verbatim}[commandchars=\\\{\}]
starting httpd help server {\ldots} done

    \end{Verbatim}

    
    
    No EGRET, o primeiro argumento será sempre eList:

    \begin{Verbatim}[commandchars=\\\{\}]
{\color{incolor}In [{\color{incolor} }]:} plot\PY{p}{(}eList\PY{p}{)}
\end{Verbatim}


    Exemplo da documentação pdf do EGRET:

\emph{Isso é para vaões!}

    \begin{Verbatim}[commandchars=\\\{\}]
{\color{incolor}In [{\color{incolor} }]:} plotFlowSingle\PY{p}{(}eList\PY{p}{,} istat\PY{o}{=}\PY{l+m}{7}\PY{p}{,}qUnit\PY{o}{=}\PY{l+s}{\PYZdq{}}\PY{l+s}{thousandCfs\PYZdq{}}\PY{p}{)}
        plotSDLogQ\PY{p}{(}eList\PY{p}{)}
        plotQTimeDaily\PY{p}{(}eList\PY{p}{,} qLower\PY{o}{=}\PY{l+m}{1}\PY{p}{,}qUnit\PY{o}{=}\PY{l+m}{3}\PY{p}{)}
        plotFour\PY{p}{(}eList\PY{p}{,} qUnit\PY{o}{=}\PY{l+m}{3}\PY{p}{)} 
        plotFourStats\PY{p}{(}eList\PY{p}{,} qUnit\PY{o}{=}\PY{l+m}{3}\PY{p}{)}
\end{Verbatim}


    Loadflex no Github:

    \begin{Verbatim}[commandchars=\\\{\}]
{\color{incolor}In [{\color{incolor} }]:} USGS\PY{o}{\PYZhy{}}R\PY{o}{/}loadflex
\end{Verbatim}


    Habilitar repositórios CRAN \& GRAN:

\emph{O repositório CRAN é de qualquer biblioteca R que seja
compreensível, então é um repositório mais aberto}

\emph{E o GRAN é a mesma coisa, com o G de ``Geological''}

    \begin{Verbatim}[commandchars=\\\{\}]
{\color{incolor}In [{\color{incolor} }]:} rprofile\PYZus{}path \PY{o}{=} \PY{k+kp}{file.path}\PY{p}{(}\PY{k+kp}{Sys.getenv}\PY{p}{(}\PY{l+s}{\PYZdq{}}\PY{l+s}{HOME\PYZdq{}}\PY{p}{)}\PY{p}{,} \PY{l+s}{\PYZdq{}}\PY{l+s}{.Rprofile\PYZdq{}}\PY{p}{)}
        \PY{k+kp}{write}\PY{p}{(}\PY{l+s}{\PYZsq{}}\PY{l+s}{\PYZbs{}noptions(repos=c(getOption(\PYZbs{}\PYZsq{}repos\PYZbs{}\PYZsq{}),}
        \PY{l+s}{    CRAN=\PYZbs{}\PYZsq{}https://cloud.r\PYZhy{}project.org\PYZbs{}\PYZsq{},}
        \PY{l+s}{    USGS=\PYZbs{}\PYZsq{}https://owi.usgs.gov/R\PYZbs{}\PYZsq{}))\PYZbs{}n\PYZsq{}}\PY{p}{,}
              rprofile\PYZus{}path\PY{p}{,} 
              append \PY{o}{=}  \PY{k+kc}{TRUE}\PY{p}{)}
\end{Verbatim}


    Instalando:

    \begin{Verbatim}[commandchars=\\\{\}]
{\color{incolor}In [{\color{incolor} }]:} install.packages\PY{p}{(}\PY{l+s}{\PYZdq{}}\PY{l+s}{loadflex\PYZdq{}}\PY{p}{)}
        \PY{k+kn}{library}\PY{p}{(}loadflex\PY{p}{)}
\end{Verbatim}


    Um blog útil: \href{owi.usgs.gov/blog}{aqui}

USGS SPARROW: Fontes poluidoras - atmosfera (despejos lineares) - solo
(despejos lineares) - despejos pontuais

Ponto de controle (monitoramento)

Mechanistic constraints + statistical features - salinidade - vazões - N
- Ph - carbono orgânico - sedimento em suspensão - coliformes -
patogênicos

Rshiny Mapper

    \hypertarget{markdown}{%
\subsubsection{Markdown}\label{markdown}}

\begin{center}\rule{0.5\linewidth}{\linethickness}\end{center}

Modo fácil de escrever código e misturar com texto formatado

\emph{Todas essas caixas de texto aqui do Jupyter Notebook estão em
Markdown!}

    \begin{Verbatim}[commandchars=\\\{\}]
{\color{incolor}In [{\color{incolor} }]:} knit
        \PY{l+s+sb}{``}`\PY{p}{\PYZob{}}r setup\PY{p}{,} include\PY{o}{=}\PY{k+kc}{FALSE}\PY{p}{\PYZcb{}}
        knitr\PY{o}{::}opts\PYZus{}chunk\PY{o}{\PYZdl{}}set\PY{p}{(}echo \PY{o}{=} \PY{k+kc}{TRUE}\PY{p}{)}
\end{Verbatim}


    \hypertarget{terceiro-dia-r-aplicado}{%
\paragraph{Terceiro dia, R aplicado}\label{terceiro-dia-r-aplicado}}

\begin{center}\rule{0.5\linewidth}{\linethickness}\end{center}

    Instalar pacotes:

    \begin{Verbatim}[commandchars=\\\{\}]
{\color{incolor}In [{\color{incolor} }]:} install.packages\PY{p}{(}\PY{k+kt}{c}\PY{p}{(}\PY{l+s}{\PYZdq{}}\PY{l+s}{ggthemes\PYZdq{}}\PY{p}{,} \PY{l+s}{\PYZdq{}}\PY{l+s}{maptools\PYZdq{}}\PY{p}{,} \PY{l+s}{\PYZdq{}}\PY{l+s}{mapdata\PYZdq{}}\PY{p}{,} \PY{l+s}{\PYZdq{}}\PY{l+s}{brazilmaps\PYZdq{}}\PY{p}{,} \PY{l+s}{\PYZdq{}}\PY{l+s}{ggmap\PYZdq{}}\PY{p}{,} \PY{l+s}{\PYZdq{}}\PY{l+s}{readx1\PYZdq{}}\PY{p}{,}
                           \PY{l+s}{\PYZdq{}}\PY{l+s}{openxlsx\PYZdq{}}\PY{p}{,} \PY{l+s}{\PYZdq{}}\PY{l+s}{sf\PYZdq{}}\PY{p}{,} \PY{l+s}{\PYZdq{}}\PY{l+s}{RPostgreSQL\PYZdq{}}\PY{p}{,} \PY{l+s}{\PYZdq{}}\PY{l+s}{toxEval\PYZdq{}}\PY{p}{,} \PY{l+s}{\PYZdq{}}\PY{l+s}{RSQLite\PYZdq{}}\PY{p}{,} \PY{l+s}{\PYZdq{}}\PY{l+s}{MonetDBLite\PYZdq{}}\PY{p}{)}\PY{p}{)}
\end{Verbatim}


    Ler o .csv e transformar em .rds:

\emph{É um formato de arquivo binário, mais compacto e eficiente do que
o clássico .csv}

    \begin{Verbatim}[commandchars=\\\{\}]
{\color{incolor}In [{\color{incolor} }]:} brazil\PYZus{}df \PY{o}{\PYZlt{}\PYZhy{}} readr\PY{o}{::}read\PYZus{}csv\PY{p}{(}\PY{l+s}{\PYZdq{}}\PY{l+s}{data/brazil\PYZus{}df.csv\PYZdq{}}\PY{p}{)}
\end{Verbatim}


    Salvar para um arquivo binário rds:

    \begin{Verbatim}[commandchars=\\\{\}]
{\color{incolor}In [{\color{incolor} }]:} \PY{k+kp}{saveRDS}\PY{p}{(}brazil\PYZus{}df\PY{p}{,} file \PY{o}{=} \PY{l+s}{\PYZdq{}}\PY{l+s}{data/brazil\PYZus{}df.rds\PYZdq{}}\PY{p}{)}
        
        brazil\PYZus{}df \PY{o}{\PYZlt{}\PYZhy{}} \PY{k+kp}{readRDS}\PY{p}{(}file \PY{o}{=} \PY{l+s}{\PYZdq{}}\PY{l+s}{data/brazil\PYZus{}df.rds\PYZdq{}}\PY{p}{)}
\end{Verbatim}


    Para testar tempos de carregamento de pacotes de dados com vários
sistemas de leitura:

    \begin{Verbatim}[commandchars=\\\{\}]
{\color{incolor}In [{\color{incolor} }]:} \PY{k+kp}{system.time}\PY{p}{(}\PY{p}{\PYZob{}}
          test\PYZus{}df \PY{o}{\PYZlt{}\PYZhy{}} read.csv\PY{p}{(}file\PY{o}{=} \PY{l+s}{\PYZdq{}}\PY{l+s}{data/test.csv\PYZdq{}}\PY{p}{)}
        \PY{p}{\PYZcb{}}\PY{p}{)}
        
        \PY{k+kp}{system.time}\PY{p}{(}\PY{p}{\PYZob{}}
          test\PYZus{}df\PYZus{}readr \PY{o}{\PYZlt{}\PYZhy{}} readr\PY{o}{::}read\PYZus{}csv\PY{p}{(}file\PY{o}{=} \PY{l+s}{\PYZdq{}}\PY{l+s}{data/test.csv\PYZdq{}}\PY{p}{)}
        \PY{p}{\PYZcb{}}\PY{p}{)}
\end{Verbatim}


    Lendo com filtro:

    \begin{Verbatim}[commandchars=\\\{\}]
{\color{incolor}In [{\color{incolor} }]:} read\PYZus{}filter \PY{o}{\PYZlt{}\PYZhy{}} \PY{k+kp}{readRDS}\PY{p}{(}file\PYZus{}name\PY{p}{)}
\end{Verbatim}


    Ler, agrupar e criar um sumário:

\emph{Existe a possibilidade de ler em um csv apenas as colunas que te
interessam, o que poupa um esforço enorme!}

    Ler e gravar em planilhas Excel:

    Sincronizar um banco de dados externo:

    \hypertarget{criar-seus-pruxf3prios-pacotes}{%
\paragraph{Criar seus próprios
pacotes}\label{criar-seus-pruxf3prios-pacotes}}

    Devtools:

    \begin{Verbatim}[commandchars=\\\{\}]
{\color{incolor}In [{\color{incolor} }]:} \PY{k+kn}{library}\PY{p}{(}devtools\PY{p}{)}
        
        
        \PY{k+kn}{library}\PY{p}{(}toxEval\PY{p}{)}
        
        explore\PYZus{}endpoints\PY{p}{(}\PY{p}{)}
\end{Verbatim}


    Mapa interativo da USGS:

    \begin{Verbatim}[commandchars=\\\{\}]
{\color{incolor}In [{\color{incolor} }]:} install\PYZus{}github\PY{p}{(}\PY{l+s}{\PYZdq{}}\PY{l+s}{USGS\PYZhy{}R/EflowStats\PYZdq{}}\PY{p}{)}
        
        \PY{o}{?}USGS\PY{o}{\PYZhy{}}R\PY{o}{/}EflowStats
        
        \PY{o}{?}\PY{o}{?}USGS\PY{o}{\PYZhy{}}R\PY{o}{/}EflowStats
        
        \PY{k+kn}{library}\PY{p}{(}USGS\PY{o}{\PYZhy{}}R\PY{o}{/}EflowStats\PY{p}{)}
\end{Verbatim}


    Mapas:

    \begin{Verbatim}[commandchars=\\\{\}]
{\color{incolor}In [{\color{incolor} }]:} install.packages \PY{p}{(}\PY{l+s}{\PYZdq{}}\PY{l+s}{brazilmaps\PYZdq{}}\PY{p}{)}
        
        \PY{k+kn}{library}\PY{p}{(}ggplot2\PY{p}{)}
        \PY{k+kn}{library}\PY{p}{(}ggthemes\PY{p}{)}
        \PY{k+kn}{library}\PY{p}{(}dplyr\PY{p}{)}
        \PY{k+kn}{library}\PY{p}{(}maptools\PY{p}{)}
        \PY{k+kn}{library}\PY{p}{(}mapdata\PY{p}{)}
        
        whole\PYZus{}br \PY{o}{\PYZlt{}\PYZhy{}} map\PY{p}{(}\PY{l+s}{\PYZsq{}}\PY{l+s}{world\PYZsq{}}\PY{p}{,} \PY{l+s}{\PYZsq{}}\PY{l+s}{brazil\PYZsq{}}\PY{p}{,} fill\PY{o}{=}\PY{k+kc}{TRUE}\PY{p}{,} plot\PY{o}{=}\PY{k+kc}{FALSE}\PY{p}{)}
        
        whole\PYZus{}br\PYZus{}df \PY{o}{\PYZlt{}\PYZhy{}} \PY{k+kt}{data.frame}\PY{p}{(}x\PY{o}{=}whole\PYZus{}br\PY{p}{[[}\PY{l+m}{1}\PY{p}{]]}\PY{p}{,} y\PY{o}{=}whole\PYZus{}br\PY{p}{[[}\PY{l+m}{2}\PY{p}{]]}\PY{p}{)}
        
        fake\PYZus{}data \PY{o}{\PYZlt{}\PYZhy{}} \PY{k+kt}{data.frame}\PY{p}{(}
          x \PY{o}{=} \PY{k+kt}{c}\PY{p}{(}\PY{l+m}{\PYZhy{}47.8}\PY{p}{,} \PY{l+m}{\PYZhy{}44.2}\PY{p}{)}\PY{p}{,}
          y \PY{o}{=} \PY{k+kt}{c}\PY{p}{(}\PY{l+m}{\PYZhy{}15.75}\PY{p}{,} \PY{l+m}{\PYZhy{}2.6}\PY{p}{)}\PY{p}{,}
          measured \PY{o}{=} \PY{k+kt}{c}\PY{p}{(}\PY{l+m}{10}\PY{p}{,} \PY{l+m}{5}\PY{p}{)}
        \PY{p}{)}
\end{Verbatim}


    Dados como um fator:

    \begin{Verbatim}[commandchars=\\\{\}]
{\color{incolor}In [{\color{incolor} }]:} fake\PYZus{}data \PY{o}{\PYZlt{}\PYZhy{}} \PY{k+kp}{as.factor}\PY{p}{(}fake\PYZus{}data\PY{o}{\PYZdl{}}measured\PY{p}{)}
        
        ggplot\PY{p}{(}\PY{p}{)} \PY{o}{+}
          geom\PYZus{}polygon\PY{p}{(}data \PY{o}{=} whole\PYZus{}br\PYZus{}df\PY{p}{,} aes\PY{p}{(}x\PY{o}{=}x\PY{p}{,}y\PY{o}{=}y\PY{p}{)}\PY{p}{,}
                        fill \PY{o}{=} \PY{l+s}{\PYZdq{}}\PY{l+s}{pink\PYZdq{}}\PY{p}{,}
                        color \PY{o}{=} \PY{l+s}{\PYZdq{}}\PY{l+s}{white\PYZdq{}}\PY{p}{)} \PY{o}{+}
          geom\PYZus{}point\PY{p}{(}data \PY{o}{=} fake\PYZus{}data\PY{p}{,}
                    aes\PY{p}{(}x\PY{o}{=}x\PY{p}{,} y\PY{o}{=}y\PY{p}{,} color\PY{o}{=}measured\PY{p}{)}\PY{p}{)}\PY{o}{+}
          coord\PYZus{}map\PY{p}{(}\PY{l+s}{\PYZdq{}}\PY{l+s}{polyconic\PYZdq{}}\PY{p}{)} \PY{o}{+}
          theme\PYZus{}map\PY{p}{(}\PY{p}{)}
        
        install.packages\PY{p}{(}\PY{l+s}{\PYZdq{}}\PY{l+s}{ggspatial\PYZdq{}}\PY{p}{)}
        
          \PY{k+kn}{library}\PY{p}{(}brazilmaps\PY{p}{)}
          \PY{k+kn}{library}\PY{p}{(}sf\PY{p}{)}
          \PY{k+kn}{library}\PY{p}{(}ggspatial\PY{p}{)}
\end{Verbatim}


    Outra versão, com o buffer para destacar o país:

    \begin{Verbatim}[commandchars=\\\{\}]
{\color{incolor}In [{\color{incolor} }]:} ggplot\PY{p}{(}\PY{p}{)} \PY{o}{+}
          geom\PYZus{}polygon\PY{p}{(}data \PY{o}{=} whole\PYZus{}br\PYZus{}df\PY{p}{,} aes\PY{p}{(}x\PY{o}{=}x\PY{p}{,}y\PY{o}{=}y\PY{p}{)}\PY{p}{,}
                        size\PY{o}{=}\PY{l+m}{10}\PY{p}{,}
                        color \PY{o}{=} \PY{l+s}{\PYZdq{}}\PY{l+s}{red\PYZdq{}}\PY{p}{)} \PY{o}{+}
          geom\PYZus{}sf\PY{p}{(}data \PY{o}{=} st\PYZus{}geometry\PY{p}{(}state\PYZus{}map\PY{p}{)}\PY{p}{,}
                   fill\PY{o}{=}\PY{l+s}{\PYZdq{}}\PY{l+s}{pink\PYZdq{}}\PY{p}{,} color\PY{o}{=}\PY{l+s}{\PYZdq{}}\PY{l+s}{blue\PYZdq{}}\PY{p}{)} \PY{o}{+}
          geom\PYZus{}point\PY{p}{(}data \PY{o}{=} fake\PYZus{}data\PY{p}{,} size \PY{o}{=} \PY{l+m}{5}\PY{p}{,}
                    aes\PY{p}{(}x\PY{o}{=}x\PY{p}{,} y\PY{o}{=}y\PY{p}{,} color\PY{o}{=}measured\PY{p}{)}\PY{p}{)}\PY{o}{+}
          coord\PYZus{}sf\PY{p}{(}datum \PY{o}{=} \PY{k+kc}{NA}\PY{p}{,} expand \PY{o}{=} \PY{k+kc}{FALSE}\PY{p}{)} \PY{o}{+}
          theme\PYZus{}map\PY{p}{(}\PY{p}{)}
        
        estado do Rio\PY{o}{:}
        \PY{c+c1}{\PYZsh{} Adding states}
          state\PYZus{}map \PY{o}{\PYZlt{}\PYZhy{}} get\PYZus{}brmap\PY{p}{(}geo \PY{o}{=} \PY{l+s}{\PYZdq{}}\PY{l+s}{State\PYZdq{}}\PY{p}{,} class \PY{o}{=} \PY{l+s}{\PYZdq{}}\PY{l+s}{sf\PYZdq{}}\PY{p}{)}
        
          rio\PYZus{}map \PY{o}{\PYZlt{}\PYZhy{}} get\PYZus{}brmap\PY{p}{(}geo \PY{o}{=} \PY{l+s}{\PYZdq{}}\PY{l+s}{State\PYZdq{}}\PY{p}{,}
                             geo.filter \PY{o}{=} \PY{k+kt}{list}\PY{p}{(}State \PY{o}{=} \PY{l+m}{33}\PY{p}{)}\PY{p}{,}
                             class \PY{o}{=} \PY{l+s}{\PYZdq{}}\PY{l+s}{sf\PYZdq{}}\PY{p}{)}
\end{Verbatim}


    \begin{Verbatim}[commandchars=\\\{\}]
{\color{incolor}In [{\color{incolor} }]:} ggplot\PY{p}{(}\PY{p}{)} \PY{o}{+}
          \PY{c+c1}{\PYZsh{}geom\PYZus{}polygon(data = whole\PYZus{}br\PYZus{}df, aes(x=x,y=y),}
          \PY{c+c1}{\PYZsh{}              size=10,}
          \PY{c+c1}{\PYZsh{}              fill = \PYZdq{}re d\PYZdq{}) +}
          geom\PYZus{}sf\PY{p}{(}data \PY{o}{=} st\PYZus{}geometry\PY{p}{(}state\PYZus{}map\PY{p}{)}\PY{p}{,}
                   fill\PY{o}{=}\PY{l+s}{\PYZdq{}}\PY{l+s}{pink\PYZdq{}}\PY{p}{,} color\PY{o}{=}\PY{l+s}{\PYZdq{}}\PY{l+s}{blue\PYZdq{}}\PY{p}{)} \PY{o}{+}
          geom\PYZus{}sf\PY{p}{(}data \PY{o}{=} st\PYZus{}geometry\PY{p}{(}rio\PYZus{}map\PY{p}{)}\PY{p}{,}
                  color\PY{o}{=}\PY{l+s}{\PYZdq{}}\PY{l+s}{red\PYZdq{}}\PY{p}{)} \PY{o}{+}
          geom\PYZus{}point\PY{p}{(}data \PY{o}{=} fake\PYZus{}data\PY{p}{,} size \PY{o}{=} \PY{l+m}{5}\PY{p}{,}
                    aes\PY{p}{(}x\PY{o}{=}x\PY{p}{,} y\PY{o}{=}y\PY{p}{,} color\PY{o}{=}measured\PY{p}{)}\PY{p}{)}\PY{o}{+}
          coord\PYZus{}sf\PY{p}{(}datum \PY{o}{=} \PY{k+kc}{NA}\PY{p}{,} expand \PY{o}{=} \PY{k+kc}{FALSE}\PY{p}{)} \PY{o}{+}
          theme\PYZus{}map\PY{p}{(}\PY{p}{)}
\end{Verbatim}


    Rios na Amazônia:

\emph{Eu preciso apontar que a geometria é a mesma do shape anterior!}

    \begin{Verbatim}[commandchars=\\\{\}]
{\color{incolor}In [{\color{incolor} }]:} amazon\PYZus{}rivers\PYZus{}shp \PY{o}{\PYZlt{}\PYZhy{}} st\PYZus{}read\PY{p}{(}\PY{l+s}{\PYZdq{}}\PY{l+s}{data/lineaire\PYZus{}10km.shp\PYZdq{}}\PY{p}{)}
        st\PYZus{}crs\PY{p}{(}amazon\PYZus{}rivers\PYZus{}shp\PY{p}{)} \PY{o}{\PYZlt{}\PYZhy{}} st\PYZus{}crs\PY{p}{(}state\PYZus{}map\PY{p}{)}
        
        clipped\PYZus{}amazon \PY{o}{\PYZlt{}\PYZhy{}} st\PYZus{}intersection\PY{p}{(}st\PYZus{}geometry\PY{p}{(}amazon\PYZus{}rivers\PYZus{}shp\PY{p}{)}\PY{p}{,}
                                          st\PYZus{}geometry\PY{p}{(}state\PYZus{}map\PY{p}{)}\PY{p}{)}
\end{Verbatim}


    \begin{Verbatim}[commandchars=\\\{\}]
{\color{incolor}In [{\color{incolor} }]:} ggplot\PY{p}{(}\PY{p}{)} \PY{o}{+}
          geom\PYZus{}polygon\PY{p}{(}data \PY{o}{=} whole\PYZus{}br\PYZus{}df\PY{p}{,} aes\PY{p}{(}x\PY{o}{=}x\PY{p}{,}y\PY{o}{=}y\PY{p}{)}\PY{p}{,}
                       size\PY{o}{=}\PY{l+m}{10}\PY{p}{,}
                       color \PY{o}{=} \PY{l+s}{\PYZdq{}}\PY{l+s}{red\PYZdq{}}\PY{p}{)} \PY{o}{+}
          geom\PYZus{}sf\PY{p}{(}data \PY{o}{=} st\PYZus{}geometry\PY{p}{(}state\PYZus{}map\PY{p}{)}\PY{p}{,}
                  fill\PY{o}{=}\PY{l+s}{\PYZdq{}}\PY{l+s}{pink\PYZdq{}}\PY{p}{,} color\PY{o}{=}\PY{l+s}{\PYZdq{}}\PY{l+s}{blue\PYZdq{}}\PY{p}{)} \PY{o}{+}
          geom\PYZus{}sf\PY{p}{(}data \PY{o}{=} st\PYZus{}geometry\PY{p}{(}rio\PYZus{}map\PY{p}{)}\PY{p}{,}
                  color\PY{o}{=}\PY{l+s}{\PYZdq{}}\PY{l+s}{red\PYZdq{}}\PY{p}{)} \PY{o}{+}
          geom\PYZus{}sf\PY{p}{(}data \PY{o}{=} clipped\PYZus{}amazon\PY{p}{,}
                  color \PY{o}{=} \PY{l+s}{\PYZdq{}}\PY{l+s}{blue\PYZdq{}}\PY{p}{)} \PY{o}{+}
          geom\PYZus{}point\PY{p}{(}data \PY{o}{=} fake\PYZus{}data\PY{p}{,} size \PY{o}{=} \PY{l+m}{5}\PY{p}{,}
                     aes\PY{p}{(}x\PY{o}{=}x\PY{p}{,} y\PY{o}{=}y\PY{p}{,} color\PY{o}{=}measured\PY{p}{)}\PY{p}{)} \PY{o}{+}
          coord\PYZus{}sf\PY{p}{(}datum \PY{o}{=} \PY{k+kc}{NA}\PY{p}{,} expand \PY{o}{=} \PY{k+kc}{FALSE}\PY{p}{)} \PY{o}{+}
          theme\PYZus{}map\PY{p}{(}\PY{p}{)}
\end{Verbatim}


    Criação de pacotes R:

\begin{itemize}
\tightlist
\item
  licença cc0 - open source
\end{itemize}

\emph{Não esqueça de dar um Install and Restart para ativar o seu novo
pacote!}

    \begin{Verbatim}[commandchars=\\\{\}]
{\color{incolor}In [{\color{incolor} }]:} \PY{k+kn}{library}\PY{p}{(}TestePacote\PY{p}{)}
        hello\PY{p}{(}\PY{l+s}{\PYZdq{}}\PY{l+s}{Eduardo\PYZdq{}}\PY{p}{)}
        \PY{o}{?}hello
\end{Verbatim}


    Em hello.R:

    \begin{Verbatim}[commandchars=\\\{\}]
{\color{incolor}In [{\color{incolor} }]:} hello \PY{o}{\PYZlt{}\PYZhy{}} \PY{k+kr}{function}\PY{p}{(}nome \PY{o}{=} \PY{l+s}{\PYZdq{}}\PY{l+s}{ANA\PYZdq{}}\PY{p}{)} \PY{p}{\PYZob{}}
          \PY{k+kp}{print}\PY{p}{(}\PY{k+kp}{paste}\PY{p}{(}\PY{l+s}{\PYZdq{}}\PY{l+s}{Hello, \PYZdq{}}\PY{p}{,} nome\PY{p}{)}\PY{p}{)}
        \PY{p}{\PYZcb{}}
\end{Verbatim}


    Em Description:

    \begin{Verbatim}[commandchars=\\\{\}]
{\color{incolor}In [{\color{incolor} }]:} Package\PY{o}{:} TestePacote
        Type\PY{o}{:} Package
        Title\PY{o}{:} Teste em pacotes R
        Version\PY{o}{:} \PY{l+m}{0.1}\PY{l+m}{.0}
        Author\PY{o}{:} Eduardo Passeto
        Maintainer\PY{o}{:} \PY{o}{\PYZlt{}}epasseto\PY{o}{@}gmail.com\PY{o}{\PYZgt{}}
        Description\PY{o}{:} Este pacote serve para demonstrar como o sistema de criar pacotes funciona
        License\PY{o}{:} What license is it under\PY{o}{?}
        \PY{k+kp}{Encoding}\PY{o}{:} UTF\PY{l+m}{\PYZhy{}8}
        LazyData\PY{o}{:} true
\end{Verbatim}


    Usar padrão Roxygen2:

\emph{Não se esqueça de ativar em ``package'' e ``configuration'',
ativando o check box!}

    \begin{Verbatim}[commandchars=\\\{\}]
{\color{incolor}In [{\color{incolor} }]:} install.packages\PY{p}{(}\PY{l+s}{\PYZdq{}}\PY{l+s}{roxygen2\PYZdq{}}\PY{p}{)}
\end{Verbatim}


    Padrão Roxygen:

\begin{verbatim}
@title ...

@description ...

@param data ...
\end{verbatim}

\emph{A importância do Roxygen é que ele mantém a documentação junto com
a função e assim ela fica sempre organizada e atualizada}

    Exemplo completo:

    \begin{Verbatim}[commandchars=\\\{\}]
{\color{incolor}In [{\color{incolor} }]:} \PY{c+c1}{\PYZsh{}\PYZsq{} @title Printing hello}
        \PY{c+c1}{\PYZsh{}\PYZsq{} @author Eduardo Passeto}
        \PY{c+c1}{\PYZsh{}\PYZsq{} @param name a strign of a name to print}
        \PY{c+c1}{\PYZsh{}\PYZsq{} @export\PYZsq{}}
        \PY{c+c1}{\PYZsh{}\PYZsq{}}
        \PY{c+c1}{\PYZsh{}\PYZsq{}}
        hello \PY{o}{\PYZlt{}\PYZhy{}} \PY{k+kr}{function}\PY{p}{(}nome \PY{o}{=} \PY{l+s}{\PYZdq{}}\PY{l+s}{ANA\PYZdq{}}\PY{p}{)} \PY{p}{\PYZob{}}
          \PY{k+kp}{print}\PY{p}{(}\PY{k+kp}{paste}\PY{p}{(}\PY{l+s}{\PYZdq{}}\PY{l+s}{Hello, \PYZdq{}}\PY{p}{,} nome\PY{p}{)}\PY{p}{)}
        \PY{p}{\PYZcb{}}
\end{Verbatim}


    Em outra janela de R, eu posso carregar meu novo pacote:

    \begin{Verbatim}[commandchars=\\\{\}]
{\color{incolor}In [{\color{incolor} }]:} \PY{k+kn}{library}\PY{p}{(}TestePacote\PY{p}{)}
\end{Verbatim}


    Para chamar minha função:

\begin{verbatim}
hello(...)
\end{verbatim}

\emph{Isso aqui aparece como origem}

\begin{verbatim}
library{testePacote}
\end{verbatim}

    Tutorial do shiny: \href{shiny.rstudio.com/tutorial}{aqui}


    % Add a bibliography block to the postdoc
    
    
    
    \end{document}
